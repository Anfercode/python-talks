% Author: Victor Terron (c) 2013
% Email: `echo vt2rron1iaa32s | tr 132 @.e`
% License: CC BY-SA 4.0

\begin{frame}[fragile]{25. for-else}
  \begin{block}{}
    \centering
    \small
    El bucle \structure{for} admite, opcionalmente, la cláusula
    \structure{else}: este bloque se ejecuta sólo si el bucle for ha
    terminado \emph{limpiamente} --- es decir, si \structure{no} se ha
    usado \structure{break} para terminar antes de tiempo.
  \end{block}

  \small
  \begin{exampleblock}
    {No usamos break, else se ejecuta}
    \begin{lstlisting}
>>> for x in range(3):
...     print x
... else:
...     print "¡Hemos terminado!"
...
0
1
2
¡Hemos terminado!
    \end{lstlisting}
  \end{exampleblock}
\end{frame}

\begin{frame}[fragile]{25. for-else}
  \small
  \begin{exampleblock}
    {Usamos break, else no se ejecuta}
    \begin{lstlisting}
>>> for x in range(3):
...     print x
...     break
... else:
...     print "¡Hemos terminado otra vez!"
...
0
    \end{lstlisting}
  \end{exampleblock}

  \begin{alertblock}{}
    \centering
    Esta construcción, for-else, nos ahorra la necesidad de usar una
    \structure{variable de bandera} para almacenar el estado de, por
    ejemplo, una búsqueda.
  \end{alertblock}
\end{frame}

\begin{frame}[fragile]{25. for-else}
  \footnotesize
  \begin{block}{}
    \centering
    Para lanzar una excepción si un elemento \structure{no ha sido
      encontrado} en una secuencia, por ejemplo, podríamos usar una
    variable que fuera \structure{False} y que, sólo si en algún
    momento encontramos el elemento a buscar, pasamos a
    \structure{True}. Al terminar el bucle, actuamos en función del
    valor de esta variable.
  \end{block}

  \begin{exampleblock}{}
    \scriptsize
    %Escape triple quotes; otherwise string is italicized
    \begin{lstlisting}[escapechar=!]
def comprueba_que_esta(x, secuencia):
    !"""! Lanza ValueError si x no está en la secuencia !"""!

    encontrado = False
    for elemento in secuencia:
    if elemento == x:
        encontrado = True
        break

    if not encontrado:
        msg = "{0} no está en {1}".format(x, secuencia)
        raise ValueError(msg)
    \end{lstlisting}
  \end{exampleblock}
\end{frame}

\begin{frame}[fragile]{25. for-else}
  \begin{block}{}
    \centering
    La cláusula \structure{else} nos permite escribirlo así:
  \end{block}

  \begin{exampleblock}{}
    \scriptsize
    \begin{lstlisting}
def comprueba_que_esta(x, secuencia):
    for elemento in secuencia:
    if elemento == x:
            break
    else:
        msg = "{0} no está en {1}".format(x, secuencia)
        raise ValueError(msg)
    \end{lstlisting}
  \end{exampleblock}
\end{frame}

\begin{frame}[fragile]{25. for-else}
  \begin{alertblock}{}
    \centering
    Los bucles \structure{while} también admiten la cláusula
    \structure{else}.
  \end{alertblock}

  \footnotesize
  Hay a quien no convence el nombre --- y no sin motivo: de
  primeras, podríamos creer que else sólo se ejecuta si el bucle for
  no llega a ejecutarse nunca. Las razones detrás de este nombre son
  históricas, por cómo los compiladores implementaban los bucles
  while.

  La forma de visualizar el else es que casi siempre está
  \structure{emparejado} con un \structure{if}: el que está dentro del
  bucle \structure{for}. Lo que estamos diciendo es "\emph{si ninguna
    de las veces el if dentro del for es cierta, entonces ejecuta el
    bloque de else}".

  \begin{block}
    {\centering Ned Batchelder: For/else}
    \centering \url{http://nedbatchelder.com/blog/201110/forelse.html}
  \end{block}
\end{frame}
