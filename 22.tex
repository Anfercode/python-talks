\begin{frame}[fragile]{22. \large Funciones lambda, y qué opina GvR}
  \small
  \begin{center}
    Antes hemos usado \structure{funciones lambda} (o anónimas) para
    definir sobre la marcha funciones sencillas que sólo necesitábamos
    una vez. Recordemos:
  \end{center}

  \footnotesize
  \begin{exampleblock}
    {Ordenar por la parte compeja:}
    \begin{lstlisting}
>>> nums = [7+9j, 4-5j, 3+2j]
>>> sorted(nums, key=lambda x: x.imag)
[(4-5j), (3+2j), (7+9j)]
    \end{lstlisting}
  \end{exampleblock}

  \small
  \begin{block}{}
    \centering
    Las funciones lambda son muy útiles y cómodas en algunas
    ocasiones.  Provienen del mundo de la \structure{programación
    funcional}, al igual que las funciones de orden superior
    \structure{map()}, \structure{filter()} y \structure{reduce()}.
  \end{block}
\end{frame}

\begin{frame}[fragile]{22. \large Funciones lambda, y qué opina GvR}
  \begin{alertblock}{}
    \large
    \centering
    Pero a Guido van Rossum no le gustan.
  \end{alertblock}

  \small
  \begin{center}
    Nunca le convencieron y, de hecho, estuvieron a punto de desaparecer en Py3K.
  \end{center}

  \footnotesize
  \begin{center}
    El nombre de lambda puede dar lugar a confusión, ya que su
    semántica \structure{no es la misma} que en otros lenguajes. La
    idea de \structure{lambda} era sólo la de servir de herramienta
    sintática para definir funciones anónimas. El problema es que
    nadie ha sido capaz todavía de encontrar una forma mejor forma
    esto, por lo que \structure{lambda} sigue entre nosotros.
   \end{center}

  \begin{block}
    {\centering Origins of Python's "Functional" Features:}
    \centering \url{http://python-history.blogspot.com.es/2009/04/origins-of-pythons-functional-features.html}
  \end{block}
\end{frame}

\begin{frame}[fragile]{22. \large Funciones lambda, y qué opina GvR}
  \small
  \begin{block}{}
    \centering
    Así que \structure{lambda} sigue existiendo, pero lo Pythónico es
    usar el módulo \structure{operator}.  Más legible, más claro, más
    explícito.
  \end{block}

  \footnotesize
  \begin{exampleblock}
    {Ordenando de nuevo por la parte compeja...}
    \begin{lstlisting}
>>> import operator
>>> nums = [7+9j, 4-5j, 3+2j]
>>> sorted(nums, key=operator.attrgetter('imag'))
[(4-5j), (3+2j), (7+9j)]
    \end{lstlisting}
  \end{exampleblock}

  \begin{exampleblock}
    {... y por el segundo elemento de la tupla:}
    \begin{lstlisting}
>>> import operator
>>> puntos = [(2, 3), (5, 1), (3, 4), (8, 2)]
>>> sorted(puntos, key=operator.itemgetter(1))
[(5, 1), (8, 2), (2, 3), (3, 4)]
    \end{lstlisting}
  \end{exampleblock}
\end{frame}
