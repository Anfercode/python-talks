% Author: Victor Terron (c) 2013
% Email: `echo vt2rron1iaa32s | tr 132 @.e`
% License: CC BY-SA 4.0

\begin{frame}[fragile]{21. La variable \_}
  \begin{alertblock}{}
    \small
    \centering
    En el modo interactivo (Python shell), la variable \structure{\_}
    siempre contiene el valor del \structure{último} \emph{statement}
    ejecutado. Es muy útil si necesitamos \structure{reusar} el último
    valor que hemos calculado y olvidamos almacenarlo en una variable:
  \end{alertblock}

  \footnotesize
  \begin{exampleblock}{}
    \begin{lstlisting}
>>> 3 * 4
12
>>> _ + 1
13
>>> _ ** 2
169
>>> range(10)
[0, 1, 2, 3, 4, 5, 6, 7, 8, 9]
>>> _[-1]
9
    \end{lstlisting}
  \end{exampleblock}
\end{frame}

\begin{frame}[fragile]{21. La variable \_}
  \begin{block}{}
    \small
    \centering
    En modo no-interactivo (un script en Python cualquiera),
    \structure{\_} es un nombre más de variable. Por convención, suele
    usarse como nombre de variable para indicar que es
    \structure{desechable} (\emph{throwaway variable}), cuando
    necesitamos una pero \structure{no llegamos a usarla}.
  \end{block}

  \footnotesize
  \begin{exampleblock}{}
    \begin{lstlisting}
>>> for _ in range(10):
...     print "¡Hola, mundo!"
    \end{lstlisting}
  \end{exampleblock}


  \begin{exampleblock}{}
    \begin{lstlisting}
>>> import os.path
>>> path = "curriculum.tex"
>>> _, ext = os.path.splitext(path)
>>> ext
'.tex'
    \end{lstlisting}
  \end{exampleblock}
\end{frame}
