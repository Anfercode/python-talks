% Author: Victor Terron (c) 2013
% Email: `echo vt2rron1iaa32s | tr 132 @.e`
% License: CC BY-SA 4.0

\begin{frame}[fragile]{18. Aplanando una lista con sum()}
  \begin{block}{}
    \centering \LARGE sum(iterable[, start])
  \end{block}

  \small
  \begin{center}
    \centering
    La función sum() acepta, opcionalmente, un segundo argumento,
    \structure{start}, cuyo valor por defecto es cero, al cual se van
    sumando los elementos de \structure{iterable}.
  \end{center}

  \footnotesize
  \begin{exampleblock}{}
    \begin{lstlisting}
>>> sum([1, 2, 3])
6
>>> sum([1, 2, 3], 10)
16
>>> sum(range(1, 10))
45
>>> sum([100], -1)
99
    \end{lstlisting}
  \end{exampleblock}
\end{frame}

\begin{frame}[fragile]{18. Aplanando una lista con sum()}
  \begin{block}{}
    \centering
    Podemos usar una lista vacía, [], como \structure{start}, para aplanar
    una lista de listas. La función sum(), al 'sumarlas', lo que hará será
    \structure{concatenarlas} una detrás de otra.
  \end{block}

  \small
  \begin{exampleblock}{}
    \begin{lstlisting}
>>> sum([[1, 2], [3], [4, 5], [6, 7, 8]], [])
[1, 2, 3, 4, 5, 6, 7, 8]
>>> sum([[4, 5], [6, 7]], range(4))
[0, 1, 2, 3, 4, 5, 6, 7]
    \end{lstlisting}
  \end{exampleblock}
\end{frame}

\begin{frame}[fragile]{18. Aplanando una lista con sum()}
  \begin{alertblock}{}
    \centering
    Es importante que \structure{start} sea una lista. De lo contrario,
    sum() intentará añadir cada una de las sublistas a un entero, cero.
    Y eso no se puede hacer.
  \end{alertblock}

  \scriptsize
  \begin{exampleblock}{}
    \begin{lstlisting}
>>> sum([[4, 5], [6, 7]])
Traceback (most recent call last):
  File "<stdin>", line 1, in <module>
TypeError: unsupported operand type(s) for +: 'int' and 'list'
    \end{lstlisting}
  \end{exampleblock}

  \small
  \begin{block}
    {\centering Stack Overflow - Flattening a list with sum():}
    \centering \url{http://stackoverflow.com/a/6632253/184363 }
  \end{block}
\end{frame}
