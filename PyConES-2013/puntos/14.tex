% Author: Victor Terron (c) 2013
% Email: `echo vt2rron1iaa32s | tr 132 @.e`
% License: CC BY-SA 4.0

\begin{frame}[fragile]
  {14. \normalsize Repeticiones innecesarias en 'if' compuestos}
  \small
  \begin{block}{}
    \centering
    A veces podemos \structure{simplificar} algunas expresiones
    lógicas. Por ejemplo, si tenemos una serie de valores y queremos
    comprobar si nuestra variable es \structure{igual a alguno de
    ellos}:
  \end{block}

  \footnotesize
  \begin{exampleblock}
    {No escribas esto:}
    \begin{lstlisting}
>> x = 4
>> x == 3 or x == 5 or x == 7:
False
    \end{lstlisting}
  \end{exampleblock}

  \begin{exampleblock}
    {Mejor hacerlo así:}
    \begin{lstlisting}
>> x in (3, 5, 7)
False
    \end{lstlisting}
  \end{exampleblock}
\end{frame}

\begin{frame}[fragile]
  {14. \normalsize Repeticiones innecesarias en 'if' compuestos}
  \begin{block}{}
    \centering
    \structure{any()} devuelve \structure{True} si al menos un
    elemento del iterable que recibe evalúa a verdadero. También
    devuelve \structure{False} si el iterable está vacío.
  \end{block}

  \begin{exampleblock}{}
    \begin{lstlisting}
>>> any([False, True, False])
True
>>> any([False, False])
False
>>> any([])
False
    \end{lstlisting}
  \end{exampleblock}
\end{frame}

\begin{frame}[fragile]
  {14. \normalsize Repeticiones innecesarias en 'if' compuestos}
  \footnotesize
  \begin{exampleblock}
    {¿Hay algún número par?}
    \begin{lstlisting}
>>> numeros = [3, 7, 8, 9]
>>> any([not x % 2 for x in numeros])
True
    \end{lstlisting}
  \end{exampleblock}

  \normalsize
  \begin{alertblock}{}
    \centering
    Aún mejor: podemos usar un \structure{generador}
  \end{alertblock}

  \footnotesize
  \begin{exampleblock}{}
    \begin{lstlisting}[escapechar=!]
>>> numeros = [3, 7, 9, 11]
>>> any!\color{red}{(}!not x % 2 for x in numeros!\color{red}{)}!
False
    \end{lstlisting}
  \end{exampleblock}

  \small
  \begin{block}
    {\centering Built-in Functions: any()}
    \centering \url{http://docs.python.org/2/library/functions.html\#any}
  \end{block}
\end{frame}

\begin{frame}[fragile]
  {14. \normalsize Repeticiones innecesarias en 'if' compuestos}
  \begin{block}{}
    \centering
    \structure{all()} devueve True si \structure{todos} los elementos
    del iterable evalúan a verdadero, o si éste está vacío.
  \end{block}

  \begin{exampleblock}{}
    \begin{lstlisting}
>>> all([False, True, False])
False
>>> all([True])
True
>>> all([])
True
    \end{lstlisting}
  \end{exampleblock}
\end{frame}

\begin{frame}[fragile]
  {14. \normalsize Repeticiones innecesarias en 'if' compuestos}
  \begin{exampleblock}
    {¿Son pares todos los números?}
    \begin{lstlisting}
>>> numeros = [1, 4, 6, 8]
>>> all(not x % 2 for x in numeros)
False
    \end{lstlisting}
  \end{exampleblock}
\end{frame}

\begin{frame}[fragile]
  {14. \normalsize Repeticiones innecesarias en 'if' compuestos}
  \begin{alertblock}{}
    \centering
    Infinitamente mejor que la alternativa de, quizás:
  \end{alertblock}

  \footnotesize
  \begin{exampleblock}{}
    \begin{lstlisting}
>>> numeros = [1, 4, 6, 8]
>>> son_pares = True
>>> for x in numeros:
...     if x % 2:
...         son_pares = False
...         break
...
>>> son_pares
False
    \end{lstlisting}
  \end{exampleblock}

  \small
  \begin{block}
    {\centering Built-in Functions: all()}
    \centering \url{http://docs.python.org/2/library/functions.html\#all}
  \end{block}
\end{frame}
