% Author: Victor Terron (c) 2013
% Email: `echo vt2rron1iaa32s | tr 132 @.e`
% License: CC BY-SA 4.0

\begin{frame}
  {29. \normalsize Implementando \_\_hash\_\_() en nuestras clases}
  \small
  \begin{block}{}
    \centering
    El método \structure{\_\_hash\_\_()} devuelve un entero que
    representa el valor hash del objeto. Es lo que se usa en los
    diccionarios, por ejemplo, para relacionar cada clave con su valor
    asociado. El único requisito es que los objetos que se consideren
    \structure{iguales} (==) han de devolver también el
    \structure{mismo valor hash}.
  \end{block}

  \begin{alertblock}{}
    \centering
    En caso de que no definamos \structure{\_\_hash\_\_()}, nuestras
    clases devuelven \structure{id()} -- un valor único para cada
    objeto, ya que es su dirección en memoria.
  \end{alertblock}
\end{frame}

\begin{frame}[fragile]
  {29. \normalsize Implementando \_\_hash\_\_() en nuestras clases}
  \tiny
  \begin{exampleblock}{}
    \begin{lstlisting}
class Punto(object):
    def __init__(self, x, y):
        self.x = x
        self.y = y

p1 = Punto(1, 7)
p2 = Punto(2, 3)
p3 = Punto(2, 3)

print "id(p1)  : ", id(p1)
print "hash(p1): ", hash(p1)
print "hash(p2): ", hash(p2)
print "hash(p3): ", hash(p3)
    \end{lstlisting}
  \end{exampleblock}

  \begin{exampleblock}
    {\scriptsize
      Esto muestra por pantalla, por ejemplo:}
    \begin{lstlisting}[escapechar=!]
id(p1)  :  140730517360144
hash(p1):  140730517360144
hash(p2):  140730516993!\color{red}{872}!
hash(p3):  140730516993!\color{red}{936}!
    \end{lstlisting}
  \end{exampleblock}
\end{frame}

\begin{frame}[fragile]
  {29. \normalsize Implementando \_\_hash\_\_() en nuestras clases}
  \footnotesize
  \begin{block}{}
    \centering
    Una forma de calcular el hash de nuestra clase, en los casos que
    no son demasiado complejos, es devolver el \structure{hash de una
      tupla que agrupa los atributos de nuestra clase}. Rápido y
    sencillo:
  \end{block}

  \scriptsize
  \begin{exampleblock} {}
    \begin{lstlisting}
class Punto(object):
    def __init__(self, x, y):
        self.x = x
        self.y = y

    def __hash__(self):
        return hash((self.x, self.y))

p1 = Punto(3, 9)
p2 = Punto(3, 9)

print "id(p1)  : ", id(p1)
print "id(p2)  : ", id(p2)
print "hash(p1): ", hash(p1)
print "hash(p2): ", hash(p2)
    \end{lstlisting}
  \end{exampleblock}
\end{frame}

\begin{frame}[fragile]
  {29. \normalsize Implementando \_\_hash\_\_() en nuestras clases}
  \small
  \begin{exampleblock}
    {Ejecutado, esto muestra:}
    \begin{lstlisting}
id(p1)  :  140050627503952
id(p2)  :  140050627504016
hash(p1):  3713083797000648531
hash(p2):  3713083797000648531
    \end{lstlisting}
  \end{exampleblock}

  \footnotesize
  \begin{block}
    {\centering Python: What's a correct and good way to implement \_\_hash\_\_()?}
    \centering \url{http://stackoverflow.com/q/2909106/184363}
  \end{block}
\end{frame}

\begin{frame}
  {29. \normalsize Implementando \_\_hash\_\_() en nuestras clases}
  \small
  \begin{alertblock}{}
    \centering
    En caso de que dos objetos devuelvan el mismo hash, aún podemos
    usarlos en diccionarios. Esto es así porque los elementos se
    indexan no sólo por su valor hash, sino que Python
    \structure{también comprueba que no son iguales} (==). Es sólo si
    son iguales en los dos sentidos que los dos objetos no pueden
    existir en el mismo diccionario.
  \end{alertblock}

  \footnotesize
  \begin{block}
    {\centering Difference between \_\_str\_\_ and \_\_repr\_\_ in Python:}
    \centering \url{http://stackoverflow.com/q/1436703/184363}
  \end{block}

  \begin{block}
    {\centering How hash collisions are resolved in Python dictionaries}
    \centering \url{http://www.shutupandship.com/2012/02/how-hash-collisions-are-resolved-in.html}
  \end{block}
\end{frame}
