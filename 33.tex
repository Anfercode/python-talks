\begin{frame}[fragile]{33. collections.namedtuple}
  \small
  \begin{block}{}
    \centering
    Devuelve un tipo de dato, subclase de \structure{tuple}, que
    podemos usar para crear tuplas que además nos permiten acceder
    también por atributos.  Con \structure{namedtuple} estamos
    definiendo nuestra propia clase en una línea de código -- no deja
    de estar bien, aunque sean clases sencillas.
  \end{block}

  \scriptsize
  \begin{exampleblock}
    {\footnotesize La clase Punto(x, y, z)}
    \begin{lstlisting}
>>> Punto = collections.namedtuple('Punto', ['x', 'y', 'z'])
>>> Punto.__mro__
(<class '__main__.Punto'>, <type 'tuple'>, <type 'object'>
    \end{lstlisting}
  \end{exampleblock}
\end{frame}

\begin{frame}[fragile]{33. collections.namedtuple}
  \small
  \begin{exampleblock}
    {Y ahora usándola:}
    \begin{lstlisting}
>>> dest = Punto(1, 4, 3)
>>> dest.x
1
>>> dest.y
4
>>> dest[2]
3
    \end{lstlisting}
  \end{exampleblock}

  \begin{block}
    {\centering Understanding Python's iterator, iterable, and iteration protocols}
    \centering \url{http://stackoverflow.com/q/9884132/184363}
  \end{block}
\end{frame}
