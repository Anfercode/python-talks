% Author: Victor Terron (c) 2014
% Email: `echo vt2rron1iaa32s | tr 132 @.e`
% License: CC BY-SA 4.0

\begin{frame}{00. Heredando de object (new-style)}
  \centering
  \small
  Hemos de heredar de \structure{object} para que nuestra clase sea
  \textit{new-style}.

  \footnotesize
  \pythoncode{./code/00/00-new-style-example.py}
  \pythonoutput{./code/00/output/00-new-style-example}

\end{frame}

\begin{frame}{00. Heredando de object: por qué}
 \begin{block}{Algo de historia}
    \centering Las clases \textit{new-style} aparecieron en Python 2.2
    (diciembre de 2001)
  \end{block}

 \begin{justify}
   Hasta Python 2.1, el concepto de clase no tenía relación con el
   tipo: los objetos de todas las (\textit{old-style}) clases tenían
   el mismo tipo: \structure{\textless type 'instance' \textgreater}.
   Las clases \textit{new-style} unifican clase y tipo — dicho de otra
   forma: una clase \textit{new-style} no es sino un tipo definido por
   el usuario.
 \end{justify}
\end{frame}

\begin{frame}{00. Heredando de object: ejemplo}
  \small
  \pythoncode{./code/00/01-old-style-example.py}
  \pythonoutput{./code/00/output/01-old-style-example}
\end{frame}

\begin{frame}{00. La vida antes de Python 2.2}
 \begin{alertblock}{}
    \centering Pero no es una cuestión sólo de qué devuelve \structure{type()}
  \end{alertblock}

  \small
  \begin{itemize}
    \item No existía \structure{super()...}
    \item ... ni los descriptores...
    \item ... ni \structure{\_\_slots\_\_}
    \item El MRO (\textit{Method Resolution Order}) era mucho más simple.
    \item Las clases podían ser lanzadas, sin heredar de \structure{Exception}.
  \end{itemize}

  \small
  \begin{block}{\centering Old style and new style classes in Python}
    \centering \url{https://stackoverflow.com/a/19950198/184363}
  \end{block}
\end{frame}

\begin{frame}{00. Lanzando una clase \textit{new-style}}
  \pythoncode{./code/00/02-new-style-raise.py}
  \vspace{0.5cm}
  \footnotesize
  \pythonoutput{./code/00/output/02-new-style-raise}
\end{frame}

\begin{frame}{00. Lanzando una clase \textit{old-style}}
  \pythoncode[fontsize=\normalsize]{./code/00/03-old-style-raise.py}
  \vspace{0.5cm}
  \footnotesize
  \pythonoutput{./code/00/output/03-old-style-raise}
\end{frame}

\begin{frame}{00. Heredando de object (new-style)}
  \begin{itemize}
    \item Por compatibilidad, las clases siguen siendo
      \textit{old-style} por defecto.
    \item En Python 3 las clases \textit{old-style} han desaparecido —
      heredar de \structure{object} es opcional.
    \item El resumen es que tenemos que heredar siempre de
      \structure{object} (o de otra clase \textit{new-style}, claro).
  \end{itemize}

  \begin{alertblock}{\centering Moraleja}
    \centering Hay que heredar de \structure{object} \emph{siempre}
  \end{alertblock}
\end{frame}
