\begin{frame}[fragile]{36. Pasando funciones a iter()}
  \begin{block}{}
    \Large
    \centering iter(function, sentinel))
  \end{block}

  \small
  \begin{alertblock}{}
    \centering
    El uso más habitual de \structure{iter()} es pasarle un único
    argumento --- el objeto sobre el que queremos iterar. Pero también
    podemos pasarle una \structure{función}, que es ejecutada una y
    otra vez: en el momento en el que uno de los valores que devuelve
    sea igual a \structure{sentinel}, nos detenemos.
  \end{alertblock}
\end{frame}

\begin{frame}[fragile]{36. Pasando funciones a iter()}
  \small
  \begin{exampleblock}
    {Lee un fichero hasta la primera línea vacía:}
    \begin{lstlisting}
with open('currículum.tex') as fd:
    for linea in iter(fd.readline, '\n'):
        print linea
    \end{lstlisting}
  \end{exampleblock}
\end{frame}

\begin{frame}[fragile]{36. Pasando funciones a iter()}
  \small
  \begin{exampleblock}
    {Genera números aleatorios hasta llegar a cero:}
    \begin{lstlisting}
import random

def aleatorio():
    return random.randint(-10, 10)

for numero in iter(aleatorio, 0):
    print numero
    \end{lstlisting}
  \end{exampleblock}
\end{frame}
