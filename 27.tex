% Author: Victor Terron (c) 2013
% Email: `echo vt2rron1iaa32s | tr 132 @.e`
% License: CC BY-SA 4.0

\begin{frame}[fragile]{27. Relanzamiento de excepciones}
  \small
  \begin{block}{}
    \centering
    \structure{raise} a secas, dentro del bloque de la cláusula
    \structure{except}, relanza la excepción con la traza (traceback)
    original intanta. Esto es muy útil si queremos hacer algo más
    antes de lanzar la excepción.
  \end{block}

  \begin{exampleblock}{}
    \scriptsize
    \begin{lstlisting}[escapechar=!]
>>> x = 1
>>> y = 0
>>> try:
...     z = x / y
... except ZeroDivisionError:
...     print "Imposible dividir. Abortando..."
...     !\color{blue}{raise}!
...
Imposible dividir. Abortando...
Traceback (most recent call last):
  File "<stdin>", line 2, in <module>
ZeroDivisionError: integer division or modulo by zero
    \end{lstlisting}
  \end{exampleblock}
\end{frame}

\begin{frame}[fragile]{27. Relanzamiento de excepciones}
  \footnotesize
  \begin{alertblock}{}
    \centering
    Pero, claro, raise relanza la \structure{última} excepción. En
    caso de que dentro del \structure{except} exista la posibilidad de
    que se lance una segunda excepción, podemos almacenarla para
    asegurarnos de que no la perdemos.
  \end{alertblock}

  \begin{exampleblock}
    {En Python 3000:}
    \tiny
    \begin{lstlisting}[escapechar=!]
import os

try:
    path = "resultado.txt"
    with open(path, "wt") as fd:
        resultado = 1 / 0
        fd.write(str(resultado))

except BaseException !\color{blue}{as e}!:

    try:
        os.unlink(path)
    except Exception as sub_e:
        print("Ha ocurrido otro error:")
        print(sub_e)

    !\color{blue}{raise e}!
    \end{lstlisting}
  \end{exampleblock}
\end{frame}

\begin{frame}[fragile]{27. Relanzamiento de excepciones}
  \footnotesize
  \begin{exampleblock}
    {Esto produce:}
    \begin{lstlisting}[escapechar=!]
>>> python3.1 relanza-excepcion.py
Ha ocurrido otro error:
[Errno 13] Permission denied: '/root/.bashrc'
Traceback (most recent call last):
  File "relanza.py", line 18, in <module>
    raise e
  File "relanza.py", line 8, in <module>
    resultado = 1 / 0
ZeroDivisionError: int division or modulo by zero
    \end{lstlisting}
  \end{exampleblock}
\end{frame}

\begin{frame}[fragile]{27. Relanzamiento de excepciones}
  \small
  \begin{block}{}
    \centering
    En Python 2 también podemos usar la notación except
    \structure{Exception, e} (sin los paréntesis), pero desde 2.6+ la
    keyword \structure{as} permite que nuestro código sea mucho menos
    ambiguo.
  \end{block}

  \begin{block}
    {\centering Python try...except comma vs 'as' in except}
    \centering \url{http://stackoverflow.com/a/2535770/184363}
  \end{block}

    \begin{block}
    {\centering PEP 3110: Catching Exceptions in Python 3000:}
    \centering \url{http://www.python.org/dev/peps/pep-3110/}
  \end{block}
\end{frame}
