% Author: Victor Terron (c) 2014
% Email: `echo vt2rron1iaa32s | tr 132 @.e`
% License: CC BY-SA 4.0

\begin{frame}{02. El primer parámetro: self}
  \begin{block}{}
    \centering
    Al definir el método de una clase, el primer parámetro ha de ser
    \structure{self}: esta variable es una referencia al objeto de la
    clase que ha llamado al método.
  \end{block}

  \begin{center}
    \small Es equivalente al \structure{this} de otros lenguajes como
    Java o C++.
  \end{center}
\end{frame}

\begin{frame}{01. Ejemplo: \_\_init\_\_() y distancia()}
  \footnotesize
  \pythoncode{./code/02/20-self-example.py}
  \vspace{0.10cm}
  \pythonoutput{./code/02/output/20-self-example}
\end{frame}

\begin{frame}{02. El nombre 'self' es una convención}
  \footnotesize
  \pythoncode{./code/02/21-this-example.py}
  \scriptsize
  \pythonoutput{./code/02/output/21-this-example}
\end{frame}

\begin{frame}{02. ¿Por qué tenemos que definirlo?}
  \begin{center}
    El parámetro no tiene por qué llamarse 'self': puede ser cualquier
    otra cosa. Pero \structure{self} es la convención, y culturalmente
    está mal visto usar algo diferente.
  \end{center}

  \begin{alertblock}{}
    \centering
    ¿Pero por qué tenemos que añadirlo explícitamente a cada método
    que definamos?  ¿Por qué no puede declararse de forma automática
    por nosotros, y así ahorrarnos esos seis caracteres?
  \end{alertblock}
\end{frame}

\begin{frame}{02. ¿Podría ser self opcional?}
  \begin{center}
    Bruce Eckel propuso en 2008 exactamente eso: que \structure{self}
    pasara a ser una keyword y pudiera usarse dentro de una clase,
    pero que no hubiera que añadirla como primer parámetro en cada
    método que definamos.
  \end{center}

  \footnotesize
  \begin{block}{\centering Self in the Argument List: Redundant is not Explicit}
    \centering \url{http://www.artima.com/weblogs/viewpost.jsp?thread=239003}
  \end{block}
\end{frame}

\begin{frame}{02. Si self fuera opcional}
  \small
  \pythoncode{./code/02/22-if-self-were-optional.py}
\end{frame}

\begin{frame}{02. Si self fuera opcional — ventajas}
  \begin{center}
    \small
    Eliminaría redundancias y errores confusos para principiantes como:
  \end{center}

  \footnotesize
  \pythoncode{./code/02/23-self-error.py}
  \vspace{0.10cm}
  \pythonoutput{./code/02/output/23-self-error}
\end{frame}

\begin{frame}{02. self \textit{no} puede ser opcional}
  \begin{block}{\centering El Zen de Python}
    \Large
    \centering
    Explícito es mejor que implícito.
  \end{block}
\end{frame}

\begin{frame}{02. La explicación larga: no es tan sencillo}
  \begin{center}
    \footnotesize
    Hay una justificación de tipo teórico: recibir el objeto actual
    como argumento refuerza la equivalencia entre dos modos de llamar
    a un método:
  \end{center}

  \scriptsize
  \pythoncode{./code/02/24-method-call-equivalency.py}
  \pythonoutput{./code/02/output/24-method-call-equivalency}
\end{frame}

\begin{frame}{02. Un argumento más práctico}
  \begin{block}{}
    \large
    \centering
    La referencia explícita a \structure{self} hace posible añadir una
    clase dinámicamente, añadiéndole un método sobre la marcha.
  \end{block}
\end{frame}

\begin{frame}{02. Añadiendo un método dinámicamente}
  \footnotesize
  \pythoncode{./code/02/25-self-add-method.py}
  \pythonoutput{./code/02/output/25-self-add-method}

  \vspace{-0.5cm}
  \begin{center}
    \structure{init\_x()} está definida fuera de la clase (es una
    función), y como argumento recibe la referencia al objeto cuyo
    atributo \structure{x} es modificado. Basado en el ejemplo de
    Guido van Rossum.
  \end{center}
\end{frame}

\begin{frame}{02. No puede ser opcional — otra razón}
  \begin{justify}
    \small
    Pero el argumento definitivo está en los \structure{decoradores}:
    al decorar una función, no es trivial determinar automáticamente
    si habría que añadirle el parámetro \structure{self} o no: el
    decorador podría estar convirtiendo la función en, en entre otros,
    un método estático (que no tiene \structure{self}) o en un método
    de clase (\structure{classmethod}, que recibe una referencia a la
    clase, no al objeto).
  \end{justify}

  \begin{alertblock}{}
    \centering
    Arreglar eso implicaría cambios no triviales en el lenguaje,
    considerar casos especiales y un poco de magia negra. Y hubiera
    roto la compatibilidad hacia atrás de Python 3.1.
  \end{alertblock}
\end{frame}

\begin{frame}{02. El primer parámetro: self}
  \footnotesize
  \begin{block}{\centering Why explicit self has to stay (Guido van Rossum)}
    \centering \url{http://neopythonic.blogspot.com.es/2008/10/why-explicit-self-has-to-stay.html}
  \end{block}

  \vspace{0.5cm}
  {
    \normalsize
    \begin{alertblock}{\centering Moraleja}
      \centering Usamos \structure{self} para acceder los atributos del objeto [*]
    \end{alertblock}
  }

  [*] Un método es un atributo ejecutable — ver \structure{callable()}
\end{frame}
