% Author: Victor Terron (c) 2014
% Email: `echo vt2rron1iaa32s | tr 132 @.e`
% License: CC BY-SA 4.0

\begin{frame}{05. Métodos estáticos}
  \begin{block}{}
    \centering
    Los método estáticos (\textit{static methods}) son aquellos que no
    necesitan acceso a \structure{ningún atributo} de ningún objeto en
    concreto de la clase.
  \end{block}
\end{frame}

\begin{frame}{05. \textit{self} innecesario — ejemplo}
  \footnotesize
  \pythoncode{./code/05/50-staticmethod-example-0.py}
  \pythonoutput{./code/05/output/50-staticmethod-example-0}
\end{frame}

\begin{frame}{05. Métodos estáticos}
  \begin{center}
    \small
    En este caso no es necesario que recibamos como primer argumento
    una referencia al objeto que está llamando el método.
  \end{center}

  \begin{block}{}
    \large
    \centering
    Podemos usar \structure{@staticmethod}, un decorador, para que
    nuestra función \structure{no} lo reciba.
  \end{block}
\end{frame}

\begin{frame}{05. \textit{self} innecesario — ejemplo}
  \scriptsize
  \pythoncode{./code/05/51-staticmethod-example-1.py}
  \pythonoutput{./code/05/output/51-staticmethod-example-1}
\end{frame}

\begin{frame}{05. Métodos estáticos}
  \begin{block}{}
    \centering
     Nada nos impediría mover este método a una función fuera de la
     clase, ya que no hace uso de ningún atributo de ningún objeto,
     pero la dejamos dentro porque su \structure{lógica} (hacer sumas)
     \structure{pertenece conceptualmente a Calculadora}.
  \end{block}

  \small
  \begin{justify}
     Los métodos estáticos no dependen de las
     \structure{características} individuales de los objetos que los
     invocan, por lo que podemos considerar que en realidad pertenecen
     "a la clase" (siempre que no los confundamos con los
     \structure{classmethods}). Eso también significa que
     \structure{los podemos llamar directamente sobre la clase}, sin
     que haya que crear un objeto de la misma.
  \end{justify}
\end{frame}

\begin{frame}{05. Métodos estáticos — sobre la clase}
  \footnotesize
  \pythoncode{./code/05/52-staticmethod-example-2.py}
  \pythonoutput{./code/05/output/52-staticmethod-example-2}
\end{frame}

\begin{frame}{05. Métodos de clase}
  \begin{block}{}
    \centering
    Los métodos de clase (\textit{class methods}) pueden visualizarse
    como una variante de los métodos normales: sí reciben un primer
    argumento, pero la referencia no es al objeto que llama al método
    (\structure{self}), sino a la \structure{clase} de dicho objeto
    (\structure{cls}, por convención).
  \end{block}
\end{frame}

\begin{frame}{05. Métodos de clase — motivación}
  \begin{alertblock}{}
    \centering
     Una referencia a la clase es lo que nos hace falta si necesitamos
     \structure{devolver otro objeto} de la misma clase.
  \end{alertblock}

  \begin{center}
    \small
    Por ejemplo, podríamos tener la clase Ameba y querer implementar
    su método fisión — que simula el proceso de bipartición y devuelve
    dos objetos de la misma clase.
  \end{center}
\end{frame}

\begin{frame}{05. Ejemplo — \textit{hard-codeando} el tipo}
  \scriptsize
  \pythoncode{./code/05/53-new-object-problem-0.py}
  \pythonoutput{./code/05/output/53-new-object-problem-0}
\end{frame}

\begin{frame}{05. Problema: \textit{hard-coding} + herencia}
  \scriptsize
  \pythoncode{./code/05/54-new-object-problem-1.py}
  \pythonoutput{./code/05/output/54-new-object-problem-1}
\end{frame}

\begin{frame}{05. Ejemplo — usando  \_\_class\_\_ o type()}
  \scriptsize
  \pythoncode{./code/05/55-new-object-problem-2.py}
  \pythonoutput{./code/05/output/55-new-object-problem-2}
\end{frame}

\begin{frame}{05. Ejemplo — más problemas}
  \begin{alertblock}{}
    \centering
    \structure{type(self)} es, de forma casi indiscutible, preferible
    a acceder a \structure{self.\_\_class\_\_}
  \end{alertblock}

  \begin{center}
    \small
    El único escenario en el que no podemos usarlo es en las clases
    \textit{old-style}, porque antes de la unificación de clase y tipo
    \structure{type()} devolvía siempre \structure{\textless type
      'instance' \textgreater}.
  \end{center}
\end{frame}

\begin{frame}{05. Ejemplo — \textit{old-style}}
  \scriptsize
  \pythoncode{./code/05/56-new-object-problem-3.py}
  \pythonoutput{./code/05/output/56-new-object-problem-3}
\end{frame}

\begin{frame}{05. Métodos de clase}
  \begin{center}
    \small
    Así que, técnicamente, \structure{\_\_class\_\_} es la única
    solución que funciona tanto en clases \textit{old-style} como
    \textit{new-style}.
  \end{center}

  \begin{alertblock}{}
    \center
    Pero todo eso da igual, porque precisamente
    \structure{@classmethod}, otro decorador, existe para recibir como
    primer argumento el tipo (es decir, la clase) del objeto que ha
    llamado al método.
  \end{alertblock}
\end{frame}

\begin{frame}{05. Métodos de clase — ejemplo}
  \scriptsize
  \pythoncode{./code/05/57-classmethod-example-0.py}
  \pythonoutput{./code/05/output/57-classmethod-example-0}
\end{frame}

\begin{frame}{05. A veces necesitamos type()}
  \begin{alertblock}{}
    \centering
    Por supuesto, usar \structure{type(self)} es inevitable cuando
    necesitamos crear objetos de la misma clase \structure{y} acceder
    a atributos del objeto que llama al método.
  \end{alertblock}
\end{frame}

\begin{frame}{05. A veces necesitamos type() — ejemplo}
  \scriptsize
  \pythoncode{./code/05/58-classmethod-example-1.py}
  \pythonoutput{./code/05/output/58-classmethod-example-1}
\end{frame}

\begin{frame}{05. Métodos de clase y estáticos}
  \small
  \begin{block}{\centering Python @classmethod and @staticmethod for beginner?}
    \centering \url{https://stackoverflow.com/a/12179752/184363}
  \end{block}

  \begin{block}{\centering Python type() or \_\_class\_\_, == or is}
    \centering \url{https://stackoverflow.com/a/9611083/184363}
  \end{block}
\end{frame}

\begin{frame}{05. Métodos de clase y estáticos}
  \begin{alertblock}{\centering Moraleja (I)}
    \centering Usamos \structure{@staticmethod} cuando el método
    trabaja con la clase, no con sus objetos.
  \end{alertblock}

  \vspace{0.25cm}
  \begin{alertblock}{\centering Moraleja (II)}
    \centering Usamos \structure{@classmethod} cuando trabajamos con
    la clase \structure{y} queremos devolver un objeto de la misma
    clase.
  \end{alertblock}
\end{frame}
