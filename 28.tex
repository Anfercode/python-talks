% Author: Victor Terron (c) 2013
% Email: `echo vt2rron1iaa32s | tr 132 @.e`
% License: CC BY-SA 4.0

\begin{frame}{28. \_\_str\_\_() vs \_\_repr\_\_()}

  \footnotesize
  \begin{alertblock}
    {\centering
      \small
      El objetivo de \_\_repr\_\_() es ser inequívoco.
    }
    \centering

    {Este método, ejecutado vía \structure{repr()}, devuelve una
      cadena de texto con la \structure{representación única del
        objeto}. Se usa sobre todo para depurar errores, por lo que la
      idea es que incluya toda la información que necesitamos --- por
      ejemplo, intentando entender qué ha fallado analizando unos
      logs.  }
  \end{alertblock}

  \begin{block}
    {\centering
      \small
      El objetivo de \_\_str\_\_() es ser legible.
    }
    \centering
    {\footnotesize
      La cadena que devuelve \structure{str()} no tiene otro fin que
      el de ser fácil de comprender por \structure{humanos}: cualquier
      cosa que aumente la legibilidad, como eliminar decimales
      inútiles o información poco importante, es aceptable.
    }
  \end{block}
\end{frame}

\begin{frame}[fragile]{28. \_\_str\_\_() vs \_\_repr\_\_()}
  \footnotesize
  \begin{exampleblock}
    {El módulo datetime nos proporciona un buen ejemplo:}
    \begin{lstlisting}
>>> import datetime
>>> today = datetime.datetime.now()
>>> str(today)
'2013-11-20 13:51:53.006588'
>>> repr(today)
'datetime.datetime(2013, 11, 20, 13, 51, 53, 6588)'
    \end{lstlisting}
  \end{exampleblock}
\end{frame}

\begin{frame}[fragile]{28. \_\_str\_\_() vs \_\_repr\_\_()}
  \small
  \begin{block}{}
    \centering
    Idealmente, la cadena devuelta por \structure{\_\_repr\_\_()}
    debería ser aquella que, pasada a \structure{eval()}, devuelve el
    mismo objeto. Al fin y al cabo, si \structure{eval()} es capaz de
    reconstruir el objeto a partir de ella, esto garantiza que
    contiene \structure{toda} la infomación que necesaria:
  \end{block}

  \footnotesize
  \begin{exampleblock} {}
    \begin{lstlisting}
>>> today = datetime.datetime.now()
>>> repr(today)
'datetime.datetime(2013, 11, 20, 13, 54, 33, 934577)'
>>> today == eval(repr(today))
True
    \end{lstlisting}
  \end{exampleblock}
\end{frame}

\begin{frame}{28. \_\_str\_\_() vs \_\_repr\_\_()}
  \begin{alertblock}{}
    \centering
    Por cierto, ¡usar \structure{eval(repr(obj))} para serializar
    objetos no es buena idea! Es poco eficiente y, mucho peor,
    peligroso. Usa \structure{pickle} mejor.
  \end{alertblock}

  \small
  \begin{block}
    {\centering Eval really is dangerous:}
    \centering \url{http://nedbatchelder.com/blog/201206/eval_really_is_dangerous.html}
  \end{block}
\end{frame}

\begin{frame}{28. \_\_str\_\_() vs \_\_repr\_\_()}
  \small
  \begin{alertblock}{}
    \centering
    \_\_repr\_\_() es para \structure{desarrolladores},
    \_\_str\_\_() para \structure{usuarios}.
  \end{alertblock}

  \begin{block}{}
    \centering
    En caso de que nuestra clase defina \_\_repr\_\_() pero no
    \_\_str\_\_(), la llamada a \structure{str() también devuelve
    repr()}. Así que el único que de verdad tenemos que
    implementar es \_\_repr\_\_().
  \end{block}

  \begin{block}
    {\centering Difference between \_\_str\_\_ and \_\_repr\_\_ in Python:}
    \centering \url{http://stackoverflow.com/q/1436703/184363}
  \end{block}
\end{frame}
