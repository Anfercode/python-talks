\begin{frame}[fragile]{16. sorted() - parámetro 'key'}
  \footnotesize
  \begin{block}{}
    \centering
    Desde Python 2.4, tanto el método \structure{list.sort()} como la
    función \structure{sorted()} aceptan el parámetro \structure{key},
    con el que se especifica la función que se ejecuta para cada
    elemento y a partir de cuyo resultado se ordena la secuencia.
  \end{block}

  \begin{exampleblock}
    {Ordenar una serie de números por su valor absoluto:}
    \begin{lstlisting}
>>> nums = [-8, 3, 5, -1, 3]
>>> sorted(nums, key=abs)
[-1, 3, 3, 5, -8]
    \end{lstlisting}
  \end{exampleblock}

  \begin{exampleblock}
    {Una lista de cadenas lexicográficamente:}
    \begin{lstlisting}
>>> gnu = ["GNU", "is", "Not", "Unix"]
>>> sorted(gnu)
['GNU', 'Not', 'Unix', 'is']
    \end{lstlisting}
  \end{exampleblock}
\end{frame}

\begin{frame}[fragile]{16. sorted() - parámetro 'key'}
  \begin{exampleblock}
    {Ahora por su longitud:}
    \begin{lstlisting}
>>> sorted(gnu, key=len)
['is', 'GNU', 'Not', 'Unix']
    \end{lstlisting}
  \end{exampleblock}

  \begin{alertblock}{}
    \centering
    Desde 2.2, las ordenaciones en Python son \structure{estables}: al
    tener 'GNU' y 'Not' la misma longitud, se mantiene el orden
    original. Esto permite ordenar en función de \structure{dos o más
    criterios}, en diferentes pasos.
  \end{alertblock}
\end{frame}

\begin{frame}[fragile]{16. sorted() - parámetro 'key'}
  \begin{block}{}
    \centering
    Podemos ordenar \structure{usando nuestras propias funciones}
  \end{block}

  \footnotesize
  \begin{exampleblock}
    {Ordenar por el número de vocales:}
    \begin{lstlisting}
>>> def nvocales(palabra):
...    contador = 0
...    for letra in palabra:
...        if letra.lower() in "aeiou":
...	   contador += 1
...    return contador
...
>>> lame = "LAME Ain't an MP3 Encoder".split()
>>> sorted(lame, key=nvocales)
['MP3', 'an', 'LAME', "Ain't", 'Encoder']
    \end{lstlisting}
  \end{exampleblock}
\end{frame}

\begin{frame}[fragile]{16. sorted() - parámetro 'key'}
  \begin{block}{}
    \centering
    O usando \structure{funciones anónimas}
  \end{block}

  \footnotesize
  \begin{exampleblock}
    {Ordenar por la parte compeja:}
    \begin{lstlisting}
>>> nums = [7+9j, 4-5j, 3+2j]
>>> sorted(nums, key=lambda x: x.imag)
[(4-5j), (3+2j), (7+9j)]
    \end{lstlisting}
  \end{exampleblock}

  \begin{exampleblock}
    {Ordenar por el segundo elemento de la tupla:}
    \begin{lstlisting}
>>> puntos = [(2, 3), (5, 1), (3, 4), (8, 2)]
>>> sorted(puntos, key=lambda x: x[1])
[(5, 1), (8, 2), (2, 3), (3, 4)]
    \end{lstlisting}
  \end{exampleblock}

  \begin{block}
    {\centering Sorting Mini-HOW TO}
    \centering \url{http://wiki.python.org/moin/HowTo/Sorting/}
  \end{block}
\end{frame}
