\begin{frame}[fragile]{3. 0.1 + 0.2 != 0.3}
  \begin{alertblock}{}
    \centering \Large ¿Por qué 0.1 + 0.2 == 0.3 es \structure{False}?
  \end{alertblock}

  \begin{exampleblock}{}
    \begin{lstlisting}
>>> 0.1 + 0.2 == 0.3
False
    \end{lstlisting}
  \end{exampleblock}

  \begin{center}
    Porque los números flotantes se representan internamente, en
    cualquier ordenador, como \structure{fracciones binarias}. Por
    ejemplo, 1.25 es equivalente a 1/2 + 3/4.
  \end{center}
\end{frame}

\begin{frame}[fragile]{3. 0.1 + 0.2 != 0.3}
  \footnotesize
  \begin{center}
    La mayor parte de las fracciones decimales no puede ser
    representada exactamente con una fracción binaria. Lo máximo que
    los números flotantes pueden hacer es \structure{aproximar su
    valor real} --- con bastantes decimales, pero nunca el exacto.
  \end{center}

  \begin{exampleblock}{}
    \small
    \begin{lstlisting}
>>> print "%.30f" % 0.1
0.100000000000000005551115123126
>>> print "%.30f" % 0.2
0.200000000000000011102230246252
>>> print "%.30f" % 0.3
0.299999999999999988897769753748
    \end{lstlisting}
  \end{exampleblock}

  \footnotesize
  \begin{center}
    Lo mismo ocurre en base decimal con, por ejemplo, el número
    1/3. Por más decimales con que se represente, 0.333333... no deja
    de ser una aproximación.
  \end{center}
\end{frame}

\begin{frame}[fragile]{3. 0.1 + 0.2 != 0.3}
  \small
  \begin{center}
    Normalmente no somos conscientes de esto porque Python por defecto
    nos muestra una aproximación del valor real de los números
    flotantes, redondeándolos a una \structure{representación
    práctica} para nosotros.
  \end{center}

  \begin{exampleblock}{}
    \small
    \begin{lstlisting}
>>> print 0.1
0.1
>>> print 0.2
0.2
>>> print 0.3
0.3
>>> 0.1 + 0.2
0.30000000000000004
    \end{lstlisting}
  \end{exampleblock}

  \small
  \begin{block}{Floating Point Arithmetic: Issues and Limitations}
    \centering \url{http://docs.python.org/2/tutorial/floatingpoint.html}
  \end{block}
\end{frame}

\begin{frame}[fragile]{3. 0.1 + 0.2 != 0.3}
  \begin{alertblock}{}
    \centering \small Nunca deberíamos comparar números flotantes
    directamente
  \end{alertblock}

  \small
  \begin{center}
    Una opción, simple, es comprobar si ambos valores son lo
    "\emph{suficientemente iguales}" usando un valor máximo de error
    absoluto admisible, normalmente llamado \structure{epsilon}.
   \end{center}

  \begin{exampleblock}{}
    \begin{lstlisting}
>>> epsilon = 1e-10
>>> abs((0.1 + 0.2) - 0.3) < epsilon
True
>>> abs((0.1 + 0.2) - 0.3)
5.5511151231257827e-17
    \end{lstlisting}
  \end{exampleblock}
\end{frame}

\begin{frame}{3. 0.1 + 0.2 != 0.3}
  \large
  \begin{center}
    En caso de no conocer el rango de los números con los que vamos a
    trabajar, tenemos que compararlos en términos de \structure{error
    relativo} \\ ("\emph{son iguales al 99.999\%}).
  \end{center}

  \small
  \begin{block}{Comparing Floating Point Numbers, 2012 Edition}
    \centering \url{http://randomascii.wordpress.com/2012/02/25/comparing-floating-point-numbers-2012-edition/}
  \end{block}
\end{frame}

\begin{frame}[fragile]{3. 0.1 + 0.2 != 0.3}
  \begin{block}{}
    \centering En Python tenemos el módulo \structure{decimal}
  \end{block}

  \begin{exampleblock}{}
    \begin{lstlisting}
>>> import decimal
>>> x = decimal.Decimal("0.1")
>>> y = decimal.Decimal("0.2")
>>> z = decimal.Decimal("0.3")
>>> x + y == z
True
    \end{lstlisting}
  \end{exampleblock}

  \small
  \begin{block}{decimal --- Decimal fixed point and floating point arithmetic}
    \centering \url{http://docs.python.org/2/library/decimal.html}
  \end{block}
\end{frame}
