% Author: Victor Terron (c) 2013
% Email: `echo vt2rron1iaa32s | tr 132 @.e`
% License: CC BY-SA 4.0

\begin{frame}[fragile]{26. try-except-else-finally}
  \begin{alertblock}{}
    \centering
    La construcción \structure{try-except-finally} admite también la
    cláusula \structure{else}: este bloque de código se ejecuta sólo
    si no se ha lanzado ninguna excepción.
  \end{alertblock}

  \begin{center}
    \small
    Esto nos permite minimizar el código que escribimos dentro de la
    cláusula \structure{try}, reduciendo el riesgo de que
    \structure{except} capture una excepción que realmente no
    queríamos proteger con el \structure{try}.
  \end{center}
\end{frame}

\begin{frame}[fragile]{26. try-except-else-finally}
  \begin{exampleblock}
    {\footnotesize
      Descarga presentación, descomprime, actualiza log}
    \tiny
    \begin{lstlisting}[escapechar=!]
import os
import urllib
import zipfile

url = "http://github.com/vterron/PyConES-2013/archive/master.zip"
path = "charla-pycon.zip"

try:
    urllib.urlretrieve(url, path)
    myzip = zipfile.ZipFile(path)
    myzip.extractall()  # puede lanzar !\color{red}{IOError}!

    # Tambien puede lanzar !\color{red}{IOError}!
    with open("descargas-log", "at") as fd:
        fd.write("{0}\n".format(path))

except !\color{red}{IOError}!:
    print "He capturado IOError!"

finally:
    os.unlink(path)
    \end{lstlisting}
  \end{exampleblock}

  \small
  \begin{alertblock}{}
    \centering
    ¿Cuál de los dos \structure{IOError} he capturado?
    \structure{No lo sabemos}
  \end{alertblock}
\end{frame}

\begin{frame}[fragile]{26. try-except-else-finally}

  \begin{exampleblock}
    {\footnotesize
      Por supuesto, podríamos evitar este problema anidando try-elses:}
    \tiny
    \begin{lstlisting}
try:
    urllib.urlretrieve(url, path)
    myzip = zipfile.ZipFile(path)
    try:
        myzip.extractall()
	try:
            with open("descargas-log", "at") as fd:
                fd.write("{0}\n".format(path))
        except IOError:
            print "Error escribiendo fichero"

    except IOError:
        print "Error descomprimiendo fichero ZIP"

finally:
    os.unlink(path)
    \end{lstlisting}
  \end{exampleblock}

  \small
  \begin{alertblock}{}
    \centering
    Pero, como dice el Zen de Python, "\emph{Flat is better than nested}"
  \end{alertblock}
\end{frame}

\begin{frame}[fragile]{26. try-except-else-finally}
  \begin{block}{}
    \centering
    Utilizando la cláusula \structure{else} basta con:
  \end{block}

  \begin{exampleblock}{}
    \footnotesize
    \begin{lstlisting}[escapechar=!]
try:
    urllib.urlretrieve(url, path)
    myzip = zipfile.ZipFile(path)
    myzip.extractall()
except IOError:
    print "Error descomprimiendo fichero ZIP"
!\color{red}{else:}!
    with open("descargas-log", "at") as fd:
        fd.write("{0}\n".format(path))
finally:
    os.unlink(path)
    \end{lstlisting}
  \end{exampleblock}
\end{frame}
