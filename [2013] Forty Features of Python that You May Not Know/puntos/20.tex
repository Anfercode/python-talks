% Author: Victor Terron (c) 2013
% Email: `echo vt2rron1iaa32s | tr 132 @.e`
% License: CC BY-SA 4.0

\begin{frame}[fragile]
  {20. \large Incremento != 1 en el operador slice}
  \begin{block}{}
    \centering \LARGE x[start:end:step]
  \end{block}

  \small
  \begin{center}
    El tercer índice, opcional, del \structure{operador slice},
    especifica el tamaño del paso --- cuánto avanzamos cada vez.
    Por defecto es uno, y por eso:
  \end{center}

  \begin{exampleblock}
    {Los tres primeros elementos}
    \begin{lstlisting}
>>> x = range(1, 10)
>>> x[:3]
[0, 1, 2]
    \end{lstlisting}
  \end{exampleblock}
\end{frame}

\begin{frame}[fragile]
  {20. \large Incremento != 1 en el operador slice}
  \begin{exampleblock}
    {Los cuatro últimos}
    \begin{lstlisting}
>>> x[-4:]
[6, 7, 8, 9]
    \end{lstlisting}
  \end{exampleblock}

  \begin{exampleblock}
    {Del segundo al quinto}
    \begin{lstlisting}
>>> x[1:5]
[1, 2, 3, 4]
    \end{lstlisting}
  \end{exampleblock}
\end{frame}

\begin{frame}[fragile]
  {20. \large Incremento != 1 en el operador slice}
  \begin{block}{}
    \large
    \centering
    Podemos usar un paso \structure{distinto de uno}
  \end{block}

  \small
  \begin{exampleblock}
    {Elementos en posiciones pares...}
    \begin{lstlisting}
>>> x[::2]
[1, 3, 5, 7, 9]
    \end{lstlisting}
  \end{exampleblock}

  \begin{exampleblock}
    {... e impares}
    \begin{lstlisting}
>>> x[1::2]
[0, 2, 4, 6, 8]
    \end{lstlisting}
  \end{exampleblock}
\end{frame}

\begin{frame}[fragile]
  {20. \large Incremento != 1 en el operador slice}
  \begin{exampleblock}
    {Recorrer los primeros nueve elementos, de tres en tres}
    \begin{lstlisting}
>>> x[:9:3]
[0, 3, 6]
    \end{lstlisting}
  \end{exampleblock}

  \begin{alertblock}{}
    \small
    \centering
    Hacia adelante, empezamos a contar en \structure{cero}, terminamos
    en \structure{n-1}
  \end{alertblock}
\end{frame}

\begin{frame}[fragile]
  {20. \large Incremento != 1 en el operador slice}
  \begin{block}{}
    \centering
    Un paso negativo permite recorrer la lista \structure{hacia atrás}
  \end{block}

  \small
  \begin{exampleblock}
    {Invierte la lista}
    \begin{lstlisting}
>>> x[::-1]
[9, 8, 7, 6, 5, 4, 3, 2, 1, 0]
    \end{lstlisting}
  \end{exampleblock}

  \begin{exampleblock}
    {Recorre la lista hacia atrás, de dos en dos}
    \begin{lstlisting}
>>> x[::-2]
[9, 7, 5, 3, 1]
    \end{lstlisting}
  \end{exampleblock}
\end{frame}

\begin{frame}[fragile]
  {20. \large Incremento != 1 en el operador slice}
  \begin{exampleblock}
    {Del último hasta el tercero, no inclusive, hacia atrás}
    \begin{lstlisting}
>>> x[:2:-1]
[9, 8, 7, 6, 5, 4, 3]
    \end{lstlisting}
  \end{exampleblock}

  \begin{alertblock}{}
    \small
    \centering
    Hacia atrás, empezamos a contar en \structure{-1}, terminamos
    en \structure{-len}
  \end{alertblock}
\end{frame}
