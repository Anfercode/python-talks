% Author: Victor Terron (c) 2014
% Email: `echo vt2rron1iaa32s | tr 132 @.e`
% License: CC BY-SA 4.0

\begin{frame}{01. Python's super() considered super!}
  \begin{block}{}
    \large
    \centering
    La función \structure{super()} nos devuelve un \structure{objeto
      proxy} que delega llamadas a una superclase.
  \end{block}

  \vspace{0.35cm}

 \begin{justify}
   En su uso más básico y común, nos permite llamar a un método
   definido en la clase base (es decir, de la que hemos heredado) sin
   tener que nombrarlas explícitamente, haciendo nuestro código más
   \structure{sencillo} y \structure{mantenible}.
 \end{justify}
\end{frame}

\begin{frame}{01. Algo que \emph{no} funciona}
  \footnotesize
  \pythoncode{./code/01/10-super-example-error.py}
  \vspace{0.10cm}
  \pythonoutput{./code/01/output/10-super-example-error}
\end{frame}

\begin{frame}{01. Hay que hacerlo así}
  \footnotesize
  \pythoncode{./code/01/11-super-example-py2k-0.py}
  \pythonoutput{./code/01/output/11-super-example-py2k-0}
\end{frame}

\begin{frame}{01. Un segundo ejemplo}
  \footnotesize
  \pythoncode{./code/01/12-super-example-py2k-1.py}
  \scriptsize
  \pythonoutput{./code/01/output/12-super-example-py2k-1}
\end{frame}

\begin{frame}{01. super() en Python 3000}

  \begin{center}
    En \structure{Python 3} ya no es necesario especificar la clase actual.
  \end{center}

  \footnotesize
  \pythoncode{./code/01/13-super-example-py3k.py}
  \pythonoutput{./code/01/output/13-super-example-py3k}
\end{frame}

\begin{frame}[fragile]{01. Herencia múltiple}
  \begin{center}
    \small
    Donde el verdadero poder de \structure{super()} se hace patente,
    sin embargo, es cuando trabajamos con herencia múltiple.
  \end{center}

  \scriptsize
  \begin{exampleblock}
    {Diagrama de clases en texto plano cutre}
    \begin{lstlisting}
     +----------+
     |  Humano  |
     +----------+
     /         \
    /           \
+--------+    +--------+
| Cyborg |    |  Ninja |
+--------+    +--------+
    \           /
     \         /
       +-----+
       |T1000|
       +-----+
    \end{lstlisting}
  \end{exampleblock}
\end{frame}

\begin{frame}{01. T-1000}
  \scriptsize
  \pythoncode{./code/01/14-multiple-inheritance-0.py}
\end{frame}

\begin{frame}{01. T-1000}
  \begin{block}{}
    \center
    Usamos \structure{super()} para llamar al método padre de T1000.
  \end{block}

  \vspace{0.50cm}

  \scriptsize
  \pythonoutput{./code/01/output/14-multiple-inheritance-0}
\end{frame}

\begin{frame}{01. T-1000}
  \begin{center}
    Hemos visto que T1000 hereda antes de Cyborg que de Ninja, por lo
    que Cyborg delega la llamada en \structure{Cyborg.ataca()}. ¿Y si
    Cyborg no define ataca()?
  \end{center}

  \vspace{0.5cm}

  \begin{alertblock}{}
    \center
    Podríamos pensar que entonces Cyborg buscará el método en su
    padre: Humano.
  \end{alertblock}
\end{frame}

\begin{frame}{01. T-1000}
  \scriptsize
  \pythoncode{./code/01/15-multiple-inheritance-1.py}
\end{frame}

\begin{frame}{01. Clases base y hermanas}
  \footnotesize
  \pythonoutput{./code/01/output/15-multiple-inheritance-1}

  \vspace{0.5cm}

  \normalsize
  \begin{block}{}
    \center
    Pero no. Lo que super() hace es recorrer el MRO y delegar en
    \structure{la primera clase que encuentra} por encima de T1000 que
    define el método que estamos buscando.
  \end{block}

  \begin{center}
    Es el \structure{hermano} de Cyborg, Ninja, quien define
    \structure{ataca()}
  \end{center}
\end{frame}

\begin{frame}{01. Superclases y herencia múltiple}
   \begin{alertblock}{}
     \Large
     \centering
     El concepto de superclase no tiene sentido en herencia múltiple.
  \end{alertblock}

  \begin{justify}
   Esto es lo terrorífico de \structure{super()}, y a su vez tan
   maravilloso: nos permite construir subclases que componen a otras
   ya existentes, determinando su comportamiento según el
   \structure{orden} en el que se heredan.
  \end{justify}
\end{frame}

\begin{frame}{01. super() — el orden importa}
  \scriptsize
  \pythoncode{./code/01/16-multiple-inheritance-2.py}
\end{frame}

\begin{frame}{01. super() — el orden importa}
  \pythonoutput{./code/01/output/16-multiple-inheritance-2}

  \footnotesize
  \begin{block}{\centering Python’s super() considered super!}
    \centering \url{https://rhettinger.wordpress.com/2011/05/26/}
  \end{block}

  \begin{block}{\centering Things to Know About Python Super [1 of 3]}
    \centering \url{http://www.artima.com/weblogs/viewpost.jsp?thread=236275}
  \end{block}

  \begin{block}{\centering Understanding Python super() and \_\_init\_\_() methods}
    \centering \url{https://stackoverflow.com/q/576169/184363}
  \end{block}
\end{frame}

\begin{frame}{01. Python's super() considered super!}
  \begin{center}
    En opinión de Michele Simionato, \structure{super()} es uno de los
    detalles de Python más complicados de entender (``\textit{one of the
      most tricky and surprising Python constructs}'')
    \end{center}

  \vspace{0.25cm}

  \begin{alertblock}{\centering Moraleja}
    \centering Usamos \structure{super()} para llamar a métodos
    definidos en alguna de las clases de las que heredamos.
  \end{alertblock}
\end{frame}
