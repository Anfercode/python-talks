% Author: Victor Terron (c) 2014
% Email: `echo vt2rron1iaa32s | tr 132 @.e`
% License: CC BY-SA 4.0

\begin{frame}{08. collections.namedtuple()}
  \begin{block}{}
    \Large
    \centering
    El módulo \structure{collections} es genial, y una de sus joyas es
    namedtuple()
  \end{block}

  \begin{center}
    \small
    Hay ocasiones en las que tenemos necesitamos clases que no son más
    que contenedores de atributos. Por ejemplo, podríamos definir la
    clase \structure{Punto} para almacenar las coordenadas (x, y, z)
    de un punto en el espacio tridimensional.
  \end{center}
\end{frame}

\begin{frame}{08. Ejemplo — sin namedtuple()}
  \footnotesize
  \pythoncode{./code/08/800-why-we-need-namedtuples-0.py}
  \pythonoutput{./code/08/output/800-why-we-need-namedtuples-0}
\end{frame}

\begin{frame}{08. collections.namedtuple()}
  \begin{alertblock}{}
    \centering
    Aunque no necesitemos hacer nada más con estos atributos, hemos
    tenido que definir un \structure{\_\_init\_\_()} lleno de
    aburridas asignaciones.
  \end{alertblock}

  \begin{center}
    Y, al no haber definido \structure{\_\_str\_\_()} ni
    \structure{\_\_repr\_\_()}, ni siquiera podemos imprimir por
    pantalla nuestro objeto de la forma que esperaríamos.
  \end{center}
\end{frame}

\begin{frame}{08. Ejemplo — sin \_\_repr\_\_()}
  \footnotesize
  \pythoncode{./code/08/801-why-we-need-namedtuples-1.py}
  \pythonoutput{./code/08/output/801-why-we-need-namedtuples-1}
\end{frame}

\begin{frame}{08. collections.namedtuple()}
  \begin{alertblock}{}
    \large
    \centering
    Tampoco podemos comparar dos puntos sin definir
    \structure{\_\_eq\_\_()}, ya que sin este método Python comparará
    posiciones en memoria.
  \end{alertblock}
\end{frame}

\begin{frame}{08. Ejemplo — sin \_\_eq\_\_()}
  \scriptsize
  \pythoncode{./code/08/802-why-we-need-namedtuples-2.py}
  \pythonoutput{./code/08/output/802-why-we-need-namedtuples-2}
\end{frame}

\begin{frame}{08. Ejemplo — con \_\_eq\_\_()}
  \footnotesize
  \pythoncode{./code/08/803-why-we-need-namedtuples-3.py}
\end{frame}

\begin{frame}{08. Esto es muy aburrido}
  \begin{block}{}
    \centering
    Para algo tan sencillo como imprimir y comparar dos puntos hemos
    tenido que definir varios métodos.
  \end{block}

  \begin{center}
    \small
    Es fácil terminar así definiendo incontables métodos mágicos
    cuando lo único que nosotros queríamos era almacenar tres
    coordenadas, nada más.
  \end{center}
\end{frame}

\begin{frame}{08. collections.namedtuple()}
  \begin{alertblock}{}
    \Large
    \centering
    Todo esto podemos ahorrárnoslo usando las
    \structure{\textit{namedtuples}}.
  \end{alertblock}
\end{frame}

\begin{frame}{08. collections.namedtuple() — Ejemplo}
  \footnotesize
  \pythoncode{./code/08/804-Point-with-namedtuple.py}
  \pythonoutput{./code/08/output/804-Point-with-namedtuple}
\end{frame}

\begin{frame}{08. collections.namedtuple()}
  \begin{block}{}
    \centering
    Lo que \structure{namedtuple()}, una \textit{factory function},
    nos devuelve, es una nueva clase que hereda de \structure{tuple}
      pero que también permite acceso a los atributos por
      \structure{nombre}.
  \end{block}

  \begin{center}
    \small
    Nada más y nada menos — pero al ser una subclase significa que
    todos los métodos mágicos que necesitábamos ya están
    implementados.
  \end{center}
\end{frame}

\begin{frame}{08. collections.namedtuple() — Ejemplo}
  \footnotesize
  \pythoncode{./code/08/805-namedtuple-explained.py}
  \pythonoutput{./code/08/output/805-namedtuple-explained}
\end{frame}

\begin{frame}{08. Creando una \textit{namedtuple}}
  \begin{block}{}
    \large
    \centering
    \structure{namedtuple()} recibe dos argumentos: el
    \structure{nombre} de la clase y sus \structure{atributos}.
  \end{block}

  \begin{center}
    \small
    Estos pueden especificarse en una secuencia de cadenas de texto, o
    en una \structure{única cadena de texto} en la que estén separados
    por espacios o comas.
  \end{center}
\end{frame}

\begin{frame}{08. Creando una \textit{namedtuple} — Ejemplo}
  \small
  \pythoncode{./code/08/806-namedtuple-field_names.py}
\end{frame}

\begin{frame}{08. Nombres de atributos}
  \begin{center}
    \small
    El nombre de los atributos puede ser cualquier identificador
    (``\textit{nombre de variable}'') válido en Python: letras,
    números o guiones bajos, \structure{pero} no pueden empezar por
    números (esto es normal) ni guión bajo (particulariad de las
    namedtuples).
  \end{center}

  \begin{block}{}
    \centering
    Tampoco puede ser ninguna \structure{keyword} de Python (como
    \structure{for}, \structure{print} o \structure{def}). De ser así,
    lanzará \structure{ValueError}.
  \end{block}
\end{frame}

\begin{frame}{08. Nombres de atributos — Ejemplo}
  \footnotesize
  \pythoncode{./code/08/807-namedtuple-invalid-identifier.py}
  \scriptsize
  \pythonoutput{./code/08/output/807-namedtuple-invalid-identifier}
\end{frame}

\begin{frame}{08. rename = True}
  \begin{block}{}
    \centering
    Podemos usar \structure{rename = True} para que los
    identificadores inválidos se renombren automáticamente,
    reemplazándolos con nombres posicionales.  También elimina
    identificadores duplicados, si los hay.
  \end{block}

  \begin{center}
    \small
    Esto es muy útil si estamos generando nuestra
    \structure{namedtuple} a partir de una consulta a una base de
    datos, por ejemplo.
  \end{center}
\end{frame}

\begin{frame}{08. rename = True — Ejemplo}
  \footnotesize
  \pythoncode{./code/08/808-namedtuple-rename.py}
  \pythonoutput{./code/08/output/808-namedtuple-rename}
\end{frame}

\begin{frame}{08. typename}
  \begin{alertblock}{}
    \centering
    Normalmente asignamos el objeto que \structure{namedtuple()} nos
    devuelve (la clase) a un objeto con el mismo nombre.
  \end{alertblock}

  \begin{center}
    \small
    Esto puede parecer redundante, pero tiene sentido: el primer
    argumento que le pasamos a \structure{namedtuple()} es el nombre
    de la clase, independientemente de qué identificador usemos en
    nuestro código para referirnos a la misma.
  \end{center}
\end{frame}

\begin{frame}{08. typename — Ejemplo}
  \footnotesize
  \pythoncode{./code/08/809-namedtuple-why-typename.py}
  \pythonoutput{./code/08/output/809-namedtuple-why-typename}
\end{frame}

\begin{frame}{08. Añadiendo métodos}
  \begin{alertblock}{}
    \centering
    Pero con las namedtuples no estamos limitados a la funcionalidad
    que nos proporcionan. Podemos \structure{heredar} de ellas para
    implementar \structure{nuestros propios métodos}.
  \end{alertblock}
\end{frame}

\begin{frame}{08. Añadiendo métodos — Ejemplo}
  \scriptsize
  \pythoncode{./code/08/810-Point-own-methods-0.py}
  \pythonoutput{./code/08/output/810-Point-own-methods-0}
\end{frame}

\begin{frame}{08. Añadiendo métodos}
  \begin{block}{}
    \centering
    Es preferible heredar de una namedtuple con el mismo nombre pero
    que \structure{comienza por un guión bajo}.
  \end{block}

  \begin{center}
    \small
    Esto es lo mismo que hace CPython cuando un módulo viene
    acompañado de una extensión en C (por ejemplo,
    \structure{\_socket}).
  \end{center}
\end{frame}

\begin{frame}{08. Añadiendo métodos — Ejemplo}
  \scriptsize
  \pythoncode{./code/08/811-Point-own-methods-1.py}
  \pythonoutput{./code/08/output/811-Point-own-methods-1}
\end{frame}

\begin{frame}{08. Añadiendo métodos}
  \begin{block}{}
   \large
   \centering
   Mejor aún, saltándonos un paso intermedio...
  \end{block}
\end{frame}

\begin{frame}{08. Añadiendo métodos — Mejor aún}
  \footnotesize
  \pythoncode{./code/08/812-Point-own-methods-2.py}
  \pythonoutput{./code/08/output/812-Point-own-methods-2}
\end{frame}

\begin{frame}{08. Un ejemplo un poco más completo}
  \scriptsize
  \pythoncode{./code/08/813-Point-own-methods-3.py}
  \pythonoutput{./code/08/output/813-Point-own-methods-3}
\end{frame}

\begin{frame}{08. verbose = True}
  \begin{block}{}
    \centering
    Al llamar a \structure{namedtuple()}, la opción
    \structure{verbose=True} hace que se nos imprima por pantalla la
    definición de la clase justo antes de construirla.
  \end{block}
\end{frame}

\begin{frame}{08. verbose = True — Ejemplo}
  \scriptsize
  \pythoncode{./code/08/814-namedtuple-verbose.py}
  \pythonoutput{./code/08/output/814-namedtuple-verbose}
\end{frame}

\begin{frame}{08. \_asdict()}
  \begin{block}{}
    \centering
    El método \structure{namedtuple.\_asdict()} nos devuelve un
    \structure{OrderedDict} que asociada cada atributo (clave del
    diccionario) a su valor correspondiente.
  \end{block}
\end{frame}

\begin{frame}{08. \_asdict() — Ejemplo}
  \small
  \pythoncode{./code/08/815-namedtuple-asdict.py}
  \pythonoutput{./code/08/output/815-namedtuple-asdict}
\end{frame}

\begin{frame}{08. \_replace()}
  \begin{center}{}
    Y por último — al heredar de \structure{tuple}, las
    \structure{namedtuples} son, por definición, inmutables. Pero no
    tenemos por qué abandonar toda esperanza.
  \end{center}

  \begin{block}{}
    \centering
    Usando el método \structure{\_replace()} podemos cambiar el valor
    de uno o más atributos, aunque lo que obtenemos es un nuevo objeto
    (¡ya que el original \structure{no puede} ser modificado!)
  \end{block}
\end{frame}

\begin{frame}{08. \_replace() — Ejemplo}
  \scriptsize
  \pythoncode{./code/08/816-namedtuple-replace.py}
  \pythonoutput{./code/08/output/816-namedtuple-replace}
\end{frame}

\begin{frame}{08. collections.namedtuple() — Ventajas}
  \begin{itemize}
    \item Proporcionan un punto de partida muy razonable.
    \item No tenemos que empezar desde cero nuestras clases.
    \item La inmutabilidad hace nuestro código más seguro.
    \item Más legibles, más elegantes.
    \item Menor uso de memoria (\structure{\_\_slots\_\_}).
  \end{itemize}
\end{frame}

\begin{frame}{08. collections.namedtuple()}
  \small
  \begin{block}
    {\centering What are ``named tuples'' in Python?}
    \centering
    \url{http://stackoverflow.com/q/2970608/184363}
  \end{block}

  \begin{block}
    {\centering namedtuple - Python Module of the Week}
    \centering
    \url{http://pymotw.com/2/collections/namedtuple.html}
  \end{block}

  \vspace{0.5cm}
  {
    \normalsize
    \begin{alertblock}{\centering Moraleja}
      \centering
      Usa \structure{namedtuple()} siempre que puedas.
    \end{alertblock}
  }
\end{frame}
