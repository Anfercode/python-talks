% Author: Victor Terron (c) 2014
% Email: `echo vt2rron1iaa32s | tr 132 @.e`
% License: CC BY-SA 4.0

\begin{frame}{07. \_\_str\_\_() y \_\_repr\_\_()}
  \begin{center}
    \small
    La documentación de Python hace referencia a que el método mágico
    \structure{\_\_str\_\_()} ha de devolver la representación
    ``informal'' del objeto, mientras que \structure{\_\_repr\_\_()}
    la ``formal''.
  \end{center}

  \begin{block}{}
    \centering
    \Large
    ¿Eso exactamente qué significa?
  \end{block}
\end{frame}

\begin{frame}{07. Imprimiendo un objeto Triangulo}
  \footnotesize
  \pythoncode{./code/07/700-neither-str-nor-repr.py}
  \pythonoutput{./code/07/output/700-neither-str-nor-repr}
\end{frame}

\begin{frame}{07. \_\_str\_\_()}
  \begin{block}{}
    \centering
    La función \structure{\_\_str\_\_()} debe devolver la
    \structure{cadena de texto} que se muestra por pantalla si
    llamamos a la función \structure{str()}.
  \end{block}

  \begin{center}
    \small
    Esto es lo que hace Python cuando usamos \structure{print}.
  \end{center}
\end{frame}

\begin{frame}{07. \_\_str\_\_() — Ejemplo}
  \scriptsize
  \pythoncode{./code/07/701-str-example-0.py}
  \pythonoutput{./code/07/output/701-str-example-0}
\end{frame}

\begin{frame}{07. \_\_str\_\_()}
  \begin{block}{}
    \centering
    No obstante, es mejor hacerlo sin \textit{hard-codear} el nombre
    de la clase, para que nuestro código sea más reusable.
  \end{block}

  \begin{center}
    Podemos acceder al atributo mágico \structure{\_\_name\_\_} de la
    clase actual, \structure{type(self)}, para obtener el nombre de la
    clase como una cadena de texto.
  \end{center}
\end{frame}

\begin{frame}{07. \_\_str\_\_() — Mejor así}
  \scriptsize
  \pythoncode{./code/07/702-str-example-1.py}
  \pythonoutput{./code/07/output/702-str-example-1}
\end{frame}

\begin{frame}{07. \_\_str\_\_()}
  \begin{alertblock}{}
    \Large
    \centering
    El objetivo de \structure{\_\_str\_\_()} es ser legible.
  \end{alertblock}

  \begin{center}
    La cadena que devuelve \structure{str()} no tiene otro fin que el
    de ser fácil de comprender \structure{por humanos}: cualquier cosa
    que aumente la legibilidad, como eliminar decimales inútiles o
    información poco importante, es aceptable.
  \end{center}
\end{frame}

\begin{frame}{07. \_\_repr\_\_()}
  \begin{alertblock}{}
    \large
    \centering
    El objetivo de \structure{\_\_repr\_\_()} es ser inequívoco.
  \end{alertblock}

  \begin{center}
    De \structure{\_\_repr\_\_()}, por el otro lado, se espera que nos
    devuelva una cadena de texto con una \structure{representación
    única} del objeto. Nos hace falta, por ejemplo, si estamos
    depurando un error y necesitamos saber, analizando unos logs, qué
    es exactamente uno de los objetos de la clase.
  \end{center}
\end{frame}

\begin{frame}{07. \_\_repr\_\_()}
  \begin{block}{}
    \large
    \centering
    A esta representación única se accede con la función
    \structure{repr()} o las comillas hacia atrás (\`{}\`{}).
  \end{block}

  \begin{center}
    \small
    Si \structure{\_\_repr\_\_()} no está definido, Python, en vez de
    dar un error, nos genera una representación automática, indicando
    el nombre de la clase y la dirección en memoria del objeto.
  \end{center}
\end{frame}

\begin{frame}{07. \_\_repr\_\_() — No definido}
  \footnotesize
  \pythoncode{./code/07/703-repr-undefined.py}
  \pythonoutput{./code/07/output/703-repr-undefined}
\end{frame}

\begin{frame}{07. \_\_repr\_\_()}
  \begin{alertblock}{}
    \centering
    Idealmente, la cadena devuelta por \structure{\_\_repr\_\_()}
    debería ser aquella que, pasada a \structure{eval()}, devuelve el
    mismo objeto.
  \end{alertblock}

  \begin{center}
    \small
    Al fin y al cabo, si \structure{eval()} es capaz de reconstruir el
    objeto a partir de ella, esto garantiza que contiene
    \structure{toda la infomación} necesaria.
  \end{center}
\end{frame}

\begin{frame}{07. \_\_repr\_\_() y eval()}
  \scriptsize
  \pythoncode{./code/07/704-repr-example.py}
  \pythonoutput{./code/07/output/704-repr-example}
\end{frame}

\begin{frame}{07. \_\_repr\_\_() sin  \_\_str\_\_()}
  \begin{block}{}
    \centering
    En caso de que nuestra clase defina \structure{\_\_repr\_\_()}
    pero no \structure{\_\_str\_\_()}, la llamada a \structure{str()}
    también devuelve \structure{\_\_repr\_\_()}.
  \end{block}
\end{frame}

\begin{frame}{07. \_\_repr\_\_() sin \_\_str\_\_() — Ejemplo}
  \footnotesize
  \pythoncode{./code/07/705-repr-no-str.py}
  \pythonoutput{./code/07/output/705-repr-no-str}
\end{frame}

\begin{frame}{07. \_\_repr\_\_() sin  \_\_str\_\_()}
  \begin{block}{}
    \centering
    Por eso en el primer ejemplo veíamos el nombre de la clase y su
    dirección en memoria: la llamada a \structure{\_\_str\_\_()}
    falló, por lo que Python nos devolvió \structure{\_\_repr\_\_()}.
  \end{block}

  \begin{alertblock}{}
    \small
    \centering
    El único que de verdad tenemos que implementar es
    \structure{\_\_repr\_\_()}.
  \end{alertblock}
\end{frame}

\begin{frame}{07. \_\_str\_\_() y \_\_repr\_\_() — Ejemplo}
  \footnotesize
  \pythoncode{./code/07/706-datetime-str-and-repr.py}
  \pythonoutput{./code/07/output/706-datetime-str-and-repr}
\end{frame}

\begin{frame}{07. \_\_str\_\_() y \_\_repr\_\_()}
  \small
  \begin{block}
    {\centering Difference between \_\_str\_\_ and \_\_repr\_\_ in Python}
    \centering
    \url{http://stackoverflow.com/q/1436703}
  \end{block}

  \vspace{0.5cm}
  {
    \normalsize
    \begin{alertblock}{\centering Moraleja}
      \centering
      \_\_repr\_\_() es para desarrolladores, \_\_str\_\_() para
      usuarios.
    \end{alertblock}
  }
\end{frame}
