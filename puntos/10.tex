% Author: Victor Terron (c) 2013
% Email: `echo vt2rron1iaa32s | tr 132 @.e`
% License: CC BY-SA 4.0

\begin{frame}[fragile]{10. str.format()}
  \small
  \begin{block}{}
    \centering
    Tanto el operador \structure{\%} como \structure{str.format()}
    permiten acceder a argumentos por su nombre, lo que es
    especialmente útil si necesitamos usar más de una vez la misma
    cadena. Pero \structure{str.format()} puede hacer muchas cosas
    más, como acceder a atributos de los argumentos, o hacer
    conversiones a str() y repr().
  \end{block}

  \footnotesize
  \begin{exampleblock}
    {Acceso a los argumentos por posición}
    \begin{lstlisting}
>>> '{0}{1}{0}'.format('abra', 'cad')
'abracadabra'
    \end{lstlisting}
  \end{exampleblock}

  \begin{exampleblock}
    {Acceso a los atributos de un argumento por posición}
    \begin{lstlisting}
>>> "Parte real: {0.real}".format(2+3j)
'Parte real: 2.0'
    \end{lstlisting}
  \end{exampleblock}
\end{frame}

\begin{frame}[fragile]{10. str.format()}
  \footnotesize
  \begin{exampleblock}
    {Acceso a argumentos por nombre}
    \scriptsize
    \begin{lstlisting}
>>> kwargs = {'nombre' : 'Pedro', 'apellido' : 'Medina'}
>>> "{nombre} {apellido} se llama {nombre}".format(kwargs)
'Pedro Medina se llama Pedro'
    \end{lstlisting}
  \end{exampleblock}

  \begin{exampleblock}
    {Acceso a atributos de argumentos por nombre}
    \scriptsize
    \begin{lstlisting}
>>> kwargs = {'numero' : 1+5j}
>>> "Parte imaginaria: {numero.imag}".format(**kwargs)
'Parte imaginaria: 5.0'
    \end{lstlisting}
  \end{exampleblock}

  \small
  \begin{block}
    {\centering Format Specification Mini-Language}
    \footnotesize
    \centering
    \url{http://docs.python.org/2/library/string.html#formatstrings}
  \end{block}
\end{frame}
