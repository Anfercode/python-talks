% Author: Victor Terron (c) 2013
% Email: `echo vt2rron1iaa32s | tr 132 @.e`
% License: CC BY-SA 4.0

\begin{frame}[fragile]
  {\large 12. Listas por comprensión vs generadores}
  \begin{block}{}
    \small
    \centering
    Las listas por comprensión nos permiten construir, \structure{en
    un único paso}, una lista donde cada elemento es el resultado de
    aplicar \structure{algunas operaciones} a todos (o algunos de) los
    miembros de otra secuencia o iterable.
  \end{block}

  \footnotesize
  \begin{exampleblock}
    {Por ejemplo, para calcular el cuadrado de los números [0, 9]:}
    \begin{lstlisting}
>>> g = [x ** 2 for x in range(10)]
[0, 1, 4, 9, 16, 25, 36, 49, 64, 81]
    \end{lstlisting}
  \end{exampleblock}

  \begin{exampleblock}
    {Esto es equivalente a:}
    \begin{lstlisting}
>>> resultado = []
>>> for x in range(10):
...     resultado.append(x ** 2)
    \end{lstlisting}
  \end{exampleblock}
\end{frame}

\begin{frame}[fragile]
  {\large 12. Listas por comprensión vs generadores}

  \begin{alertblock}{}
    \centering
    El problema, a veces, es que las listas por comprensión
    construyen... pues eso, una lista, almacenando \structure{todos
    los valores en memoria} a la vez.
  \end{alertblock}

  \footnotesize
  \begin{exampleblock}
    {No intentéis esto en casa}
    \begin{lstlisting}
>>> [x ** 2 for x in xrange(1000000000000000)]
# Tu ordenador explota
    \end{lstlisting}
  \end{exampleblock}
\end{frame}

\begin{frame}[fragile]
  {\large 12. Listas por comprensión vs generadores}
  \small
  \begin{block}{}
    \centering
    Si sólo necesitamos los valores una vez, y de uno en uno, podemos
    utilizar una \structure{expresión generadora}: son idénticos a las
    listas por comprensión, pero usando \structure{paréntesis}.
  \end{block}

  \footnotesize
  \begin{exampleblock}{}
    \begin{lstlisting}
>>> cuadrados = (x ** 2 for x in range(1, 11))
>>> cuadrados
<generator object <genexpr> at 0x7f8b82504eb0>
>>> next(cuadrados)
1
>>> next(cuadrados)
4
    \end{lstlisting}
  \end{exampleblock}

  \small
  \begin{block}
    {\centering PEP 289: Generator Expressions}
    \centering \url{http://www.python.org/dev/peps/pep-0289/}
  \end{block}
\end{frame}

\begin{frame}[fragile]
  {\large 12. Listas por comprensión vs generadores}
  \footnotesize
  \begin{exampleblock}
    {Equivalente a:}
    \begin{lstlisting}
>>> def cuadrados(seq):
...     for x in seq:
...         yield x ** 2
...
>>> g = cuadrados(range(1, 11))
>>> next(g)
1
>>> next(g)
4
    \end{lstlisting}
  \end{exampleblock}

  \small
  \begin{block}
    {\centering Generator Tricks for Systems Programmers (David M. Beazley)}
    \centering \url{http://www.dabeaz.com/generators/}
  \end{block}
\end{frame}
