% Author: Victor Terron (c) 2013
% Email: `echo vt2rron1iaa32s | tr 132 @.e`
% License: CC BY-SA 4.0

\begin{frame}[fragile]{39. Límite de memoria}
  \small
  \begin{block}{}
    \centering
    Antes hemos mencionado que crear una lista de tamaño excesivo
    podría usar \structure{toda la memoria} del ordenador, a veces con
    trágicas consecuencias. Usando esto podemos impedir que Python use
    más del 3/4 de la memoria total de nuestro equipo, abortando la
    ejecución \structure{si sobrepasamos este límite}.
  \end{block}

  \scriptsize
  \begin{exampleblock}{}
    \begin{lstlisting}
def get_ram_size():
    with open('/proc/meminfo', 'rt') as fd:
        regexp = "^MemTotal:\s*(\d+) kB$"
        match = re.match(regexp, fd.read(), re.MULTILINE)
        return int(match.group(1)) * 1024 # kB to bytes
MAX_RAM = get_ram_size() * 0.75
resource.setrlimit(resource.RLIMIT_AS,
                   (MAX_RAM, resource.RLIM_INFINITY))
    \end{lstlisting}
  \end{exampleblock}
\end{frame}

\begin{frame}[fragile]{39. Límite de memoria}
  \footnotesize
  \begin{exampleblock}
    {Esto es lo que ocurre entonces si abarcamos demasiado:}
    \begin{lstlisting}
>>> list(xrange(10e12))
Traceback (most recent call last):
  File ``<stdin>'', line 1, in <module>
MemoryError
    \end{lstlisting}
  \end{exampleblock}

  \small
  \begin{block}
    {\centering Por si alguien quiere reusar mi fichero .pythonrc}
    \centering \url{http://github.com/vterron/dotfiles}
  \end{block}
\end{frame}
