% Author: Victor Terron (c) 2013
% Email: `echo vt2rron1iaa32s | tr 132 @.e`
% License: CC BY-SA 4.0

\begin{frame}[fragile]{9. printf en Python}
  \small
  \begin{block}{}
    \centering
    Los objetos str y unicode tienen el \structure{operador \%}, que
    nos permite usar la sintaxis del antediluviano, presente en todos
    los lenguajes, \structure{printf()}, para controlar exactamente
    cómo se muestra una cadena de texto.
  \end{block}

  \footnotesize
  \begin{exampleblock}{}
    \begin{lstlisting}
>>> print math.pi
3.14159265359
>>> "%.2f" % math.pi
3.14
>>> "%05.2f" % math.pi
03.14
>>> potencia = math.e ** 100
>>> "%.2f ^ %.2f = %.4g" % (math.e, 100, potencia)
'2.72 ^ 100.00 = 2.688e+43'
    \end{lstlisting}
  \end{exampleblock}
\end{frame}

\begin{frame}[fragile]{9. printf en Python}
  \footnotesize
  \begin{exampleblock}{}
    \begin{lstlisting}
>>> "%s por %d es %f" % ("tres", 2, 6.0)
'tres por 2 es 6.000000'
>>> "%0*d" % (5, 3)
'00003'
>>> "%+.*f" % (5, math.pi)
'+3.14159'
    \end{lstlisting}
  \end{exampleblock}

  \begin{alertblock}{}
    \centering
    \small
    No obstante, desde Python 2.6 lo recomendable es usar
    \structure{str.format()}: más sofisticado, flexible, extensible y
    que puede trabajar con tuplas y diccionarios de forma natural. ¡El
    operador de \% para el formateo de cadenas se considera
    \structure{obsoleto}!
  \end{alertblock}

  \begin{block}
    {\centering PEP 3101: Advanced String Formatting [2006]}
    \centering \url{http://www.python.org/dev/peps/pep-3101/}
  \end{block}
\end{frame}
