% Author: Victor Terron (c) 2013
% Email: `echo vt2rron1iaa32s | tr 132 @.e`
% License: CC BY-SA 4.0

\begin{frame}[fragile]{17. Por qué "python" > [130, 129]}
  \small
  \begin{block}{}
    \centering
    Listas y tuplas se ordenan \structure{lexicográficamente},
    comparando uno a uno los elementos. Para ser iguales, ambas
    secuencias han de tener el mismo \structure{tipo}, la misma
    \structure{longitud} y sus elementos ser \structure{iguales}.
  \end{block}

  \begin{exampleblock}{}
    \begin{lstlisting}
>>> [1, 2] == [1, 2]
True
>>> [1, 2] == [3, 4]
False
>>> [1, 2] == (1, 2)
False
    \end{lstlisting}
  \end{exampleblock}
\end{frame}

\begin{frame}[fragile]{17. Por qué "python" > [130, 129]}
  \small
  \begin{alertblock}{}
    \centering
    En caso de \structure{no} tener el mismo número de elementos, la
    comparación de las dos secuencias está determinada por el primer
    elemento en el que sus valores diferen.
  \end{alertblock}

  \begin{exampleblock}{}
    \begin{lstlisting}
>>> (4, 2, 5) > (1, 3, 2)
True
>>> [2, 3, 1] >= [2, 4, 2]
False
    \end{lstlisting}
  \end{exampleblock}
\end{frame}

\begin{frame}[fragile]{17. Por qué "python" > [130, 129]}
  \small
  \begin{block}{}
    \centering
    Si el elemento correspondiente, donde los valores difieren, no
    existe, \structure{la secuencia más corta} va antes.
  \end{block}

  \begin{exampleblock}{}
    \begin{lstlisting}
>>> (2, 3) > (1,2,3)
True
>>> [1, 2] < [1, 2, 3]
True
    \end{lstlisting}
  \end{exampleblock}
\end{frame}

\begin{frame}[fragile]{17. Por qué "python" > [130, 129]}
  \small
  \begin{alertblock}{}
    \centering
    Los objetos de tipos diferentes nunca se consideran iguales, y se
    ordenan de forma \structure{consistente pero arbitraria}. En
    CPython, el criterio escogido es el del nombre de sus tipos.
  \end{alertblock}

  \footnotesize
  \begin{exampleblock}{}
    \begin{lstlisting}[escapechar=!]
>>> "python" > [130, 129]    # "str" > "list"
True
>>> {1 : "uno"} > [1, 2, 3]  # "dict" < "list"
False
>>> (23,) > [34, 1]          # "tuple" > "list"
True
    \end{lstlisting}
  \end{exampleblock}

  \begin{block}
    {\centering Stack Overflow: Maximum of two tuples}
    \centering \url{http://stackoverflow.com/a/9576006/184363}
  \end{block}
\end{frame}

\begin{frame}[fragile]{17. Por qué "python" > [130, 129]}
  \begin{block}{}
    \centering
    Pero un tipo de dato \structure{numérico} va siempre antes si lo
    comparamos con uno \structure{no numérico}.
  \end{block}

  \footnotesize
  \begin{exampleblock}{}
    \begin{lstlisting}
>>> 23 < "Python"
True
>>> 23 < (3, 1, 4)
True
>>> 23 > [3, 1, 4]
False
>>> 23 < {"pi" : 3.1415}
True
    \end{lstlisting}
  \end{exampleblock}

  \begin{block}
    {\centering Stack Overflow: How does Python compare string and int?}
    \centering \url{http://stackoverflow.com/a/3270689/184363}
  \end{block}
\end{frame}

\begin{frame}[fragile]{17. Por qué "python" > [130, 129]}
  \begin{alertblock}{}
    \centering
    Esto es un detalle de la \structure{implementación} de
    \structure{CPython}.
  \end{alertblock}

  \small
  \begin{center}
    Nuestro código nunca debería depender de que valores de diferente
    tipos se ordenen así. Y, en cualquier caso, ¿qué haces comparando
    valores de diferente tipo? ¿Por qué querrías hacer eso?
  \end{center}

  \begin{exampleblock}
    {\centering En Py3K el comportamiento ya es el que esperaríamos:}
    \begin{lstlisting}
>>> 23 < "Python"
Traceback (most recent call last):
  File "<stdin>", line 1, in <module>
TypeError: unorderable types: int() < str()
    \end{lstlisting}
  \end{exampleblock}
\end{frame}
