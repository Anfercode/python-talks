% Author: Victor Terron (c) 2013
% Email: `echo vt2rron1iaa32s | tr 132 @.e`
% License: CC BY-SA 4.0

\begin{frame}{2. \large Encadenamiento de operadores lógicos}
  \begin{alertblock}{En vez de escribir}
    \centering \LARGE x => y and y < z
  \end{alertblock}

  \small
  \begin{center}
    A diferencia de C, todos los operadores lógicos tienen la misma
    prioridad, y pueden ser encadenados de forma arbitraria.
  \end{center}

  \begin{block}{Mucho mejor}
    \centering \LARGE x <= y < z
  \end{block}

  \small
  \begin{center}
    Las dos expresiones de arriba son equivalentes, aunque en la
    segunda \structure{x} sólo se evalúa una vez. En ambos casos,
    \structure{z} no llega a evaluarse si no se cumple que
    \structure{x <= y} (\emph{lazy evaluation}).
  \end{center}
\end{frame}
