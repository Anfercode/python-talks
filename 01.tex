\begin{frame}[fragile]{1. Intercambiar dos variables}

  \begin{center}
    Normalmente, en otros lenguajes de programación, tenemos que usar
    una \structure{variable temporal} para almacenar uno de los dos
    valores.
  \end{center}

  \small
  \begin{exampleblock}{Por ejemplo, en C}
    \begin{lstlisting}[language=C]
int x, y;
int tmp;
tmp = x;
x = y;
y = x;
    \end{lstlisting}
  \end{exampleblock}
\end{frame}

\begin{frame}[fragile]{1. Intercambiar dos variables}
  \begin{block}{Python nos permite hacer}
    \centering \LARGE a, b = b, a
  \end{block}

  \begin{exampleblock}{}
    \begin{lstlisting}
>>> a = 5
>>> b = 7
>>> a, b = b, a
>>> a
7
>>> b
5
    \end{lstlisting}
  \end{exampleblock}
\end{frame}

\begin{frame}{1. Intercambiar dos variables}
  \begin{alertblock}{}
    \centering Desde nuestro punto de vista, ambas asignaciones
    ocurren simultáneamente. La clave está en que tanto \\
    \structure{a, b} como \structure{b, a} son \structure{tuplas}.
  \end{alertblock}

  Las expresiones en Python se evalúan de izquierda a derecha. En una
  asignación, el lado derecho se evalúa antes que el derecho. Por tanto:

  \begin{itemize}
    \item El lado derecho de la asignación es evaluado, creando una
      \structure{tupla de dos elementos} en memoria, cuyos elementos
      son los objetos designados por los identificadores \structure{b}
      y \structure{a}.
  \end{itemize}
\end{frame}

\begin{frame}{1. Intercambiar dos variables}
  \begin{itemize}
    \item El lado izquierdo es evaluado: Python ve que estamos
      asignando una tupla de dos elementos a otra tupla de dos
      elementos, así que \emph{desempaqueta} (\structure{tuple
        unpack}) la tupla y los asigna uno a uno:
      \begin{itemize}
        \item Al primer elemento de la tupla de la izquierda,
          \structure{a}, le asigna el primer elemento de la tupla que
          se ha creado en memoria a la derecha, el objeto que antes
          tenía el identificador \structure{b}. Así, el nuevo
          \structure{a} es el antiguo \structure{b}.
        \item Del mismo modo, el nuevo \structure{b} pasa a ser el
          antiguo \structure{a}.
      \end{itemize}
  \end{itemize}

  \small
  \begin{block}{\centering Explicación detallada en Stack Overflow:}
    \centering \url{http://stackoverflow.com/a/14836456/184363}
  \end{block}
\end{frame}

\begin{frame}{1. Intercambiar dos variables}
  \begin{center}
    ¿Cómo se declara una tupla de \structure{un único elemento}?
  \end{center}

  \begin{block}{}
    \centering \huge 1,
  \end{block}

  o, para más claridad,

  \begin{block}{}
    \centering \huge (1,)
  \end{block}

  \begin{alertblock}{}
    \centering Es la \structure{coma}, no el paréntesis, el
    constructor de la tupla
  \end{alertblock}
\end{frame}

\begin{frame}[fragile]{1. Intercambiar dos variables}

  \small
  \begin{block}{}
    \centering Los paréntesis por sí mismos no crean una tupla: Python
    \structure{evalúa la expresión} dentro de los mismos y devuelve el
    valor resultante:
  \end{block}

  \small
  \begin{exampleblock}{Esto sólo suma dos números}
    \begin{lstlisting}
>>> 2 + (1)
3
    \end{lstlisting}
  \end{exampleblock}

  \begin{exampleblock}{Esto intenta sumar entero y tupla (y fracasa)} \scriptsize
    \begin{lstlisting}
>>> 2 + (1,)
Traceback (most recent call last):
  File "<stdin>", line 1, in <module>
TypeError: unsupported operand type(s) for +: 'int' and 'tuple'
    \end{lstlisting}
  \end{exampleblock}
\end{frame}
