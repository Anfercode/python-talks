% Author: Victor Terron (c) 2013
% Email: `echo vt2rron1iaa32s | tr 132 @.e`
% License: CC BY-SA 4.0

\begin{frame}[fragile]{24. *args y **kwargs}
  \begin{block}{}
    \small
    \centering
    Nuestras funciones pueden recibir un \structure{número variable}
    de argumentos.
  \end{block}

  \begin{exampleblock}{}
    \footnotesize
    \begin{lstlisting}
>>> def imprime_punto2D(x, y):
...     print "({0}, {1})".format(x, y)
...
>>> def imprime_punto3D(x, y, z):
...     print "({0}, {1}, {2})".format(x, y, z)
...
>>> imprime_punto2D(1, 2)
(1, 2)
>>> imprime_punto3D(4, 3, 3)
(4, 3, 3)
    \end{lstlisting}
  \end{exampleblock}
\end{frame}

\begin{frame}[fragile]{24. *args y **kwargs}
  \begin{block}{}
    \small
    \centering
    Podríamos generalizarlo a cualquier número de dimensiones:
  \end{block}

  \begin{exampleblock}{}
    \scriptsize
    \begin{lstlisting}
>>> def imprime_punto(coords):
...     print "({0})".format(", ".join(str(x) for x in coords))
...
>>> imprime_punto([1, 2])
(1, 2)
>>> imprime_punto([1, 2, 3, 4])
(1, 2, 3, 4)
    \end{lstlisting}
  \end{exampleblock}

  \begin{alertblock}{}
    \small
    \centering
    Pero esto nos obliga a pasar las coordenadas \structure{en una
      secuencia}.
  \end{alertblock}
\end{frame}

\begin{frame}[fragile]{24. *args y **kwargs}
  \begin{block}{}
    \small
    \centering
    Añadiendo \structure{un asterisco} al nombre del parámetro cuando
    definimos la función podemos hacer que Python los
    \structure{agrupe} todos a partir de ese punto en una tupla.
  \end{block}

  \begin{exampleblock}{}
    \scriptsize
    \begin{lstlisting}
>>> def imprime_punto(*coords):
...     print "coordenadas: ", coords
...     print "({0})".format(", ".join(str(x) for x in coords))
...
>>> imprime_punto(2, 4)
coordenadas: (2, 4)
(2, 4)
>>> imprime_punto(1, 3, 2)
coordenadas: (1, 3, 2)
(1, 3, 2)
    \end{lstlisting}
  \end{exampleblock}
\end{frame}

\begin{frame}[fragile]{24. *args y **kwargs}
  \begin{block}{}
    \centering
    Los parámetros listados \structure{antes} del que tiene el
    asterisco se comportan normalmente.
  \end{block}

  \begin{exampleblock}{}
    \scriptsize
    \begin{lstlisting}
>>> def imprime(x, y, *resto):
...     print "x:", x
...     print "y:", y
...     print "resto:", resto
...
>>> imprime(1, 2, 3, 7)
x: 1
y: 2
resto: (3, 7)
>>> imprime(5, 3)
x: 5
y: 3
resto: ()
    \end{lstlisting}
  \end{exampleblock}
\end{frame}

\begin{frame}[fragile]{24. *args y **kwargs}
  \begin{block}{}
    \small
    \centering
    A la inversa, podemos usar el asterisco al llamar una función para
    \structure{desempaquetar} una sequencia y asignar sus elementos,
    uno a uno, a los diferentes parámetros que la función recibe.
  \end{block}

  \begin{exampleblock}{}
    \scriptsize
    \begin{lstlisting}
>>> def imprime_punto2D(x, y):
...     print "({0}, {1})".format(x, y)
...
coords = (4, 5)
>>> imprime_punto2D(coords)
Traceback (most recent call last):
  File "<stdin>", line 1, in <module>
TypeError: imprime_punto2D() takes exactly 2 arguments (1 given)
>>> imprime_punto2D(*coords)
(4, 5)
    \end{lstlisting}
  \end{exampleblock}
\end{frame}

\begin{frame}[fragile]{24. *args y **kwargs}
  \begin{alertblock}{}
    \small
    \centering
    Lo mismo puede hacerse, pero usando el \structure{doble
      asterisco}, con los argumentos nombrados (\emph{keyword
      arguments}): Python \structure{agrupa} todos los que recibe en
    un diccionario.
  \end{alertblock}

  \begin{exampleblock}{}
    \scriptsize
    \begin{lstlisting}
>>> def imprime_punto(**coordenadas):
...     print "coordenadas: ", coordenadas
...     for clave, valor in coordenadas.items():
...         print "{0} = {1}".format(clave, valor)
...
>>> imprime
_punto(x = 4, y = 8)
coordenadas:  {'y': 8, 'x': 4}
y = 8
x = 4
>>> imprime_punto(z = 3, x = 7, y = 2)
coordenadas:  {'y': 2, 'x': 7, 'z': 3}
y = 2
x = 7
z = 3
    \end{lstlisting}
  \end{exampleblock}
\end{frame}

\begin{frame}[fragile]{24. *args y **kwargs}
  \begin{block}{}
    \small
    \centering
    Y, también, a la hora de llamar a una función podemos usar
    \structure{**} para \structure{desempaquetar} un diccionario y
    pasar cada uno de sus elementos como un argumento nombrado.
  \end{block}

  \begin{exampleblock}{}
    \scriptsize
    \begin{lstlisting}
>>> def imprime_punto2D(x = 0, y = 0):
...     print "x = {0}".format(x)
...     print "y = {0}".format(y)
...
>>> coords = {'x' : 1}
>>> imprime_punto2D(**coords)
x = 1
y = 0
>>> coords = {'x' : 3, 'y' : 7}
>>> imprime_punto2D(**coords)
x = 3
y = 7
    \end{lstlisting}
  \end{exampleblock}
\end{frame}

\begin{frame}[fragile]{24. *args y **kwargs}
  \begin{block}{}
    \small
    \centering
    Es habitual usar \structure{*args} y \structure{**kwargs} como
    nombres de variables cuando definimos una función que recibe un
    número variable de parámetros o parámetros nombrados,
    respectivamente.
  \end{block}

  \footnotesize
  \begin{block}
    {\centering
      En GiHub hay 710K+ ejemplos en los que se usa esa notación:}
    \centering
    \url{https://github.com/search?l=python&q=*args\%2C+**kwargs&ref=searchresults&type=Code}
  \end{block}

\begin{block}
    {\centering
      Magnus Lie Hetland — Beginning Python \\
      Collecting Parameters (capítulo 6, "Abstraction")
    }
    \centering \url{http://hetland.org/writing/beginning-python-2/}
  \end{block}
\end{frame}
