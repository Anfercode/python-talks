% Author: Victor Terron (c) 2013
% Email: `echo vt2rron1iaa32s | tr 132 @.e`
% License: CC BY-SA 4.0

\begin{frame}[fragile]{8. Concatenación eficiente de cadenas}
  \small
  \begin{exampleblock}
    {Tenemos varias cadenas de texto que queremos unir:}
    \small
    \begin{lstlisting}
>>> palabras = "uno", "dos", "tres"
>>> resultado = ""
>>> for p in palabras:
>>>     resultado += p
>>> resultado
'unodostres'
    \end{lstlisting}
  \end{exampleblock}

  \begin{block}{}
    Las cadenas de texto en Python son \structure{inmutables}, por lo
    que cada vez que asignamos una cadena a una variable \structure{un
    nuevo objeto} es creado en memoria. En este código estamos
    calculando, almacenando y desechando cada paso intermedio.
    Eso es inaceptablemente lento.
  \end{block}
\end{frame}

\begin{frame}[fragile]{8. Concatenación eficiente de cadenas}
  \begin{alertblock}
    {\centering \small
      La forma Pythónica de concatenar cadenas es así:}
    \centering \Large resultado = "".join(palabras)
  \end{alertblock}

  \small
  \begin{center}
    El método \structure{str.join(iterable)} devuelve una cadena que
    es la concatenación de las cadenas en \structure{iterable},
    utilizando como separador entre los elementos la cadena que llama
    al método.
  \end{center}

  \begin{exampleblock}{}
    \begin{lstlisting}
>>> palabras = "uno", "dos", "tres"
>>> "-".join(palabras)
'uno-dos-tres'
    \end{lstlisting}
  \end{exampleblock}

  \small
  \begin{center}
    Y lo hace \structure{de una única pasada} --- rápido y eficiente.
  \end{center}
\end{frame}
