\begin{frame}[fragile]{31. dict.\_\_missing\_\_()}
  \small
  \begin{block}{}
    \centering
    En las \structure{subclases} de dict que definen
    \structure{\_\_missing\_\_()}, el acceso a claves que no estén
    presentes en el diccionario devuelve lo que este método devuelva.
  \end{block}

  \scriptsize
  \begin{exampleblock}{}
    % Escape triple quotes; otherwise string is italicized
    \begin{lstlisting}[escapechar=!]
class MyDict(dict):
    !"""! Diccionario que devuelve -1 si la clave no existe !"""!

    def __missing__(self, key):
        print "'{0}' no encontrado".format(key)
        return -1

>>> d = MyDict()
>>> d[1] = 3
>>> d[1]
3
>>> d[5]
'5' no encontrado
-1
    \end{lstlisting}
  \end{exampleblock}
\end{frame}

\begin{frame}[fragile]{31. dict.\_\_missing\_\_()}
  \small
  \begin{alertblock}{}
    \centering
    \_\_missing\_\_() \structure{no puede ser una variable} -- ha de ser un
    método, que recibirá como argumento la clave que no se ha
    encontrado.
  \end{alertblock}

  \scriptsize
  \begin{exampleblock}
    {\footnotesize Algo que no va a funcionar:}
    \begin{lstlisting}[escapechar=!]
class MyDict(dict):
    !"""! Diccionario que devuelve -1 si la clave no existe !"""!

    __missing__ = -1

>>> d = MyDict()
>>> d[7]
Traceback (most recent call last):
  File "<stdin>", line 1, in <module>
TypeError: 'int' object is not callable
    \end{lstlisting}
  \end{exampleblock}
\end{frame}
